\documentclass{article}
\usepackage[utf8]{inputenc}
\usepackage[T1]{fontenc}
\usepackage{CJKutf8}

\usepackage{multirow}
\usepackage{multicol}
\usepackage{listings}

\usepackage{geometry}
\geometry{b5paper}


\title{CS572 Project \# Report}
\author{Heyan Huang}

\usepackage{geometry}
\geometry{left=0cm,right=0cm,top=0.1cm,bottom=0cm}

\newenvironment{narrow}[2]{% 
\begin{list}{}{% 
\setlength{\topsep}{0pt}% 
\setlength{\leftmargin}{#1}% 
\setlength{\rightmargin}{#2}% 
%\setlength{\listparindent}{\parindent}% 
%\setlength{\itemindent}{\parindent}% 
\setlength{\parsep}{\parskip}% 
}% 
\item[]}{\end{list}} 
%\begin{narrow}{0.35in}{0in}  \end{narrow}
\begin{document}
\begin{CJK}{UTF8}{gbsn}
\maketitle


\lstset{language=c++,
numbers=left, 
numberstyle=\tiny, 
%keywordstyle=\color{blue!70}, commentstyle=\color{red!55!green!55!blue!55}, 
%frame=shadowbox, 
%rulesepcolor=\color{red!20!green!20!blue!20},
escapeinside=``, 
%xleftmargin=0em,xrightmargin=0em, aboveskip=0.5em
extendedchars=false %这一条命令可以解决代码跨页时,章节标题,页眉等汉字不显示的问题
}


\section{Algorithm Descriptions}
\subsection{Hill Climbing}
   Hill Climbing is a mathematical optimization technique which belong to the family of local search. It starts from a random solution to a problem and trying to find a better solutions by incrementally change one element of the solution. Hill Climbing is simply a greedy searching algorithm. It selects the best solution from nearby tried selection space, and does this step repeatedly until eventually it finds a best optimum. 

The advantage for this algorithm is that it is simple and easy, but it usually ends up with local optimum only, and misses out the global optimum because it always accepts only the best solutions.
\begin{description}
\item Data Structure: 
  \begin{itemize}
    \itemsep=-3pt
  \item In this project, I have used double precision float variables. 
  \item I used a vector of integer of length 15 to record the mutated dimentions, and
  \item a vector of double (of length 30) to record the mutated point. 
  \end{itemize}

\item Step Descriptions
  \begin{itemize}
    \itemsep=-3pt
  \item Random Start Point: In this project, for each functions, I started with randomly generated 30-dimention point. 
  \item Generate Neighbor: The points are 30-dimentional. I generated neighbors by mutate 50\% of the dimentions of the current point. The mutation is simply increase or decrease a small amount of value (within the range of (0, 1)) based on current point dimention. 
  \item Fitness Calculation: The fitness functions are listed out as descripted and listed below. They are deterministic. 
  \item Accept or Reject: Besed on the comparson of fitness values between current point and mutated point, we accepts the mutation if the fitness is lower then current fitness value. Otherwise, we keep unchanged and try to find other neighbor which could be potentially better than current one. 
  \item Update Current Loop Result: If we accept the mutation, we update the current point values of all the dimentions; otherwise, we keep the current point the same.
  \end{itemize}

\item Detail Attention: 
  \begin{itemize}
    \itemsep=-3pt
  \item Check the range of the mutated dimentions. Each mutated dimention must be within the range of function difinition. 
  \item At end of each mutation, after having updated decision results, we need to clean up the loop temporatory values, like the vector of integer to record the mutated dimention indexes, the vector of doulbe used to record mutated point. 
  \end{itemize}
\end{description}

The hill climbing algorithm pseudo-code for Schwefel function is pasted below: 
\begin{lstlisting}[language=c++]
Continuous Space Hill Climbing Algorithm
Loop until no changes happen any more
    Generate a random solution S
    loop do
        Fitness(S) = EVAL(S);
        Generate 15 random integer within [0, 29] used as mutated dimention index
        Generate random double [0.0, 1.0] as changes added to those 15 dimentions
        Calculate mutated point S'
            Fitness(S') = EVAL(S');
            if ( Fitness(S') < Fitness(S) )
                S <-- S'
                Fitness(S) <-- Fitness(S')
End loop
return S;
\end{lstlisting}
\begin{lstlisting}[language=c++]
\end{lstlisting}

\subsection{Simulated Annealing}
Simulated anealing is a generic metaheuristic for the global optimization problem of locating a good approximation to the global optimum of a given function in a large search space. 

Compared with hill climbing, instead of always selecting the best neighbor, Simulated Annealing does the same accept the neighbor when the Neighbor is already performs better than current position. While Simulated Annealing also accepts worse neighbor with some probability when the currently mutated worse Neighbor has a good chance potentially leading to a better global optimum. This probability is a function of Annealing temperature. As the temperature reduced down, the probability of accepting worse Neighbor is also reduced down.

Since Simulated Annealing accepts worse Neighbors when necessary, and in the end of loop I may not have found the global optimum yet, I would better use a vector of double type to record the best point. And later on, I could restart from this best point.  

\begin{description}
\item Data Structure: 
I have used about the same data Structure as used in hill climbing.
  \begin{itemize}
    \itemsep=-3pt
  \item In this project, I have used double precision float variables to record mutated fitness and best fitness. 
  \item I used a vector of integer of length 15 to record the mutated dimentions, 
  \item a vector of double (of length 30) to record the best point that mimimize the function, and
  \item another vector of double (of length 30) to record the mutated point. 
  \end{itemize}

\item Step Descriptions
  \begin{itemize}
    \itemsep=-3pt
  \item Random Start Point: In this project, for each functions, I started with randomly generated 30-dimention point. 
  \item Generate Neighbor: The points are 30-dimentional. I generated neighbors by mutate 50\% of the dimentions of the current point. The mutation is simply increase or decrease a small amount of value (within the range of (0, 1)) based on current point dimention. 
  \item Fitness Calculation: The fitness functions are listed out as descripted and listed below. They are deterministic. 
  \item Accept or Reject: Besed on the comparson of fitness values between current point and mutated point, we accepts the mutation if the fitness is lower then current fitness value already. Otherwise, we accept changes only with ceitain probability when the mutated point has good chance leading to potentially better global optimum.
  \item Update Current Loop Result: If we accept the mutation, we update the current point values of all the dimentions; otherwise, we keep the current point the same.
  \end{itemize}

\item Detail Attention: 
  \begin{itemize}
    \itemsep=-3pt
  \item Check the range of the mutated dimentions. Each mutated dimention must be within the range of function difinition. 
  \item Since we may give up better current point Compared with mutated one, when we meet better solutions, we need to update the best fitness value and the best point vector. 
  \item At end of each mutation, after having updated decision results, we need to clean up the loop temporatory values, like the vector of integer to record the mutated dimention indexes, the vector of doulbe used to record mutated point. 
  \end{itemize}
\end{description}

Thesimulated Annealing algorithm pseudo-code for Schwefel function is pasted below: 
\begin{lstlisting}[language=c++]
Start with a Random point S
for T = 100 to 0 step -1:
     Fitness(S) = EVAL(S)
     Pick a Neighbor of S, S'
     Fitness(S') = EVAL(S')
     if ( Fitness(S') < Fitness(S) )
         S <-- S'
     else
         With probability P( Fitness(S), Fitness(S'), T)
             S <-- S'
return S
\end{lstlisting}


\section{Results}
The hill climbing Algorithm supposed to find the global optimum, here listed the hill climbing results for both functions. 

\begin{table}[!hbp]  
\begin{tabular}{|c|c|c|c|c|c|}   
\hline %绘制一条水平的线
\hline %再绘制一条水平的线
name        & beginning        & Endding            & Beginning     & Endding \\
            & Sphere           & Sphere             & Schwefel      & Schwefel  \\
\hline
x[0]	&	-4.5000	&	-4.5000	&	-412.0200	&	  0.0000	\\
\hline
x[1]	&	1.3000	&	1.3000	&	435.0000	&	435.0000	\\
\hline
x[2]	&	0.4500	&	0.4500	&	47.0000  	&        47.0000	\\
\hline
x[3]	&	2.0550	&	2.0550	&	511.0000	&	511.0000	\\
\hline
x[4]	&	2.0120	&	2.0120	&	176.0000	&	  0.0000	\\
\hline
x[5]	&	-3.5000	&	-4.0500	&	-112.0000	&      -112.0000	\\
\hline
x[6]	&	2.3000	&	2.3000	&	235.0000	&	235.0000	\\
\hline
x[7]	&	2.4500	&	2.4500	&	447.0000	&	447.0000	\\
\hline
x[8]	&	-3.0550	&	-3.0550	&	509.0000	&	0.0000	\\
\hline
x[9]	&	5.0120	&	-4.0500	&	476.0000	&	476.0000	\\
\hline
x[10]	&	-2.5000	&	-2.5000	&	-512.0000	&	-512.0000	\\
\hline
x[11]	&	3.3000	&	3.3000	&	135.0000	&	135.0000	\\
\hline
x[12]	&	0.4500	&	0.4500	&	347.0000	&	347.0000	\\
\hline
x[13]	&	-1.0550	&	-1.0550	&	209.0000	&	209.0000	\\
\hline
x[14]	&	-5.0120	&	-5.0120	&	511.9600	&	511.9600	\\
\hline
x[15]	&	-1.5000	&	-1.5000	&	-312.0000	&	-312.0000	\\
\hline
x[16]	&	4.3000	&	4.3000	&	335.0000	&	335.0000	\\
\hline
x[17]	&	0.4500	&	0.4500	&	147.0000	&	147.0000	\\
\hline
x[18]	&	5.0550	&	5.0550	&	309.0000	&	309.0000	\\
\hline
x[19]	&	4.0120	&	4.0120	&	376.0000	&	376.0000	\\
\hline
x[20]	&	5.0500	&	5.0500	&	-412.0000	&	-412.0000	\\
\hline
x[21]	&	-5.0000	&	-5.0000	&	435.0000	&	435.0000	\\
\hline
x[22]	&	4.4500	&	4.4500	&	247.0000	&	247.0000	\\
\hline
x[23]	&	1.0550	&	1.0550	&	509.0000	&	509.0000	\\
\hline
x[24]	&	3.0120	&	3.0120	&	276.0000	&	276.0000	\\
\hline
x[25]	&	-4.5000	&	-4.5000	&	311.0000	&	311.0000	\\
\hline
x[26]	&	-5.1200	&	-5.1200	&	135.0000	&	135.0000	\\
\hline
x[27]	&	-5.0500	&	-5.0500	&	397.0000	&	397.0000	\\
\hline
x[28]	&	3.0550	&	3.0550	&	409.0000	&	409.0000	\\
\hline
x[29]	&	-1.0120	&	-1.0120	&	476.0000	&	476.0000	\\
\hline
\hline
Fitness &  357.39091400 &  352.82577000 & 15885.60153515        & 15654.51787996        \\
\hline
\hline
\end{tabular}
\caption{Hill Climbing Algorithm Results} 
\end{table}

\begin{table}[!hbp]  
\begin{tabular}{|c|c|c|c|c|c|}   
\hline %绘制一条水平的线
\hline %再绘制一条水平的线
name        & beginning        & Endding            & Beginning     & Endding \\
            & Sphere           & Sphere             & Schwefel      & Schwefel  \\
\hline
x[0]	&	-4.5000	&	-4.5000	&	-412.0200	&	-412.0200	\\
\hline
x[1]	&	 1.3000	&	 1.3000	&	 435.0000	&	 435.0000	\\
\hline
x[2]	&	 0.4500	&	 0.4500	&	  47.0000	&	  47.0000	\\
\hline
x[3]	&	 2.0550	&	 2.0550	&	 511.0000	&	   0.0000	\\
\hline
x[4]	&	 2.0120	&	 2.0120	&	 176.0000	&	   0.0000	\\
\hline
x[5]	&	-3.5000	&	-4.0500	&	-112.0000	&	-112.0000	\\
\hline
x[6]	&	 2.3000	&	 2.3000	&	 235.0000	&	 235.0000	\\
\hline
x[7]	&	 2.4500	&	 2.4500	&	 447.0000	&	 447.0000	\\
\hline
x[8]	&	-3.0550	&	-3.0550	&	 509.0000	&	   0.0000	\\
\hline
x[9]	&	 5.0120	&	-4.0500	&	 476.0000	&	   0.0000	\\
\hline
x[10]	&	-2.5000	&	-2.5000	&	-512.0000	&	-512.0041	\\
\hline
x[11]	&	 3.3000	&	 3.3000	&	 135.0000	&	 135.0000	\\
\hline
x[12]	&	 0.4500	&	 0.4500	&	 347.0000	&	 347.0000	\\
\hline
x[13]	&	-1.0550	&	-1.0550	&	 209.0000	&	 209.0000	\\
\hline
x[14]	&	-5.0120	&	-5.0120	&	 511.9600	&	 511.9681	\\
\hline
x[15]	&	-1.5000	&	-1.4999	&	-312.0000	&	 511.9681	\\
\hline
x[16]	&	 4.3000	&	 4.3000	&	  35.0000	&	 335.0000	\\
\hline
x[17]	&	 0.4500	&	 0.4500	&	 147.0000	&	 147.0000	\\
\hline
x[18]	&	 5.0550	&	 5.0550	&	 309.0000	&	 309.0000	\\
\hline
x[19]	&	 4.0120	&	 4.0120	&	 376.0000	&	 376.0000	\\
\hline
x[20]	&	 5.0500	&	 5.0500	&	-412.0000	&	 511.9681	\\
\hline
x[21]	&	-5.0000	&	-5.0000	&	 435.0000	&	 435.0000	\\
\hline
x[22]	&	 4.4500	&	 4.4500	&	 247.0000	&	 511.9681	\\
\hline
x[23]	&	 1.0550	&	 1.0550	&	 509.0000	&	 509.0000	\\
\hline
x[24]	&	 3.0120	&	 3.0115	&	 276.0000	&	 511.9681	\\
\hline
x[25]	&	-4.5000	&	-4.5000	&	 311.0000	&	 311.0000	\\
\hline
x[26]	&	-5.1200	&	-5.1200	&	 135.0000	&	 135.0000	\\
\hline
x[27]	&	-5.0500	&	-5.0500	&	 397.0000	&	 397.0000	\\
\hline
x[28]	&	 3.0550	&	 3.0550	&	 409.0000	&	 409.0000	\\
\hline
x[29]	&	-1.0120	&	-1.0120	&	 476.0000	&	 476.0000	\\
\hline
\hline
Fitness &  357.39091400 &  350.82548000 &  15885.60153515        & 15546.72769305        \\
\hline
\hline
\end{tabular}
\caption{Simulated Annealing Algorithm Results} 
\end{table}


\section{Conclusions}
Hill climbing is supposed to find the local optimum, and Simulated Annealing should be able to find the global optimum. The results from both methods and both functions indicate that Simulated Annealing Algorithm does find better solutions than hill climbing. But due to time and parameter selection, the Simulated Annealing Algorithm has not been able to find the global optimum yet. 

The possible reasons that Simulated Annealing failed to find the global optimum is that: 
\begin{itemize}
\itemsep=-3pt
\item I selected the Neighbors completely random, though with 15 dimentions slight changes. The randomness make lose the built ground. So to find the global optimum takes time. 

\item The completely randomness of selecting Neighbors also produces trouble. Though we have tried to restrict the range to be 15 dimentions each mutate point, and within [0, 1] chages to each dimension value. There are still very big searching space. 

\item For these two questions, since both of them are sparable, one possible solution is to try to find the global optimum for the Simulated Annealing Algorithm separately for each dimention, and then combine the globally optimal dimension solutions together to get the global optimum. 
\end{itemize}

\end{CJK}
\end{document}
