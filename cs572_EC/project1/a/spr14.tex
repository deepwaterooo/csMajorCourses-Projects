\documentclass{article}  %开始文档
\usepackage{multirow}    %使用多栏宏包

\title{Class Schedule Spring 2014}
\author{Heyan Huang}
\date{ }

\begin{document}         %开始文档
%\maketitle

Week: \#\ \ \ \ \ \ \ \ \  Date: 0\ \ /\ \ \ \ /2014 
\begin{table}[!hbp]   %开始表格
%其中参数[!hbp] 的意思是:
%!表示尽可能的尝试 h(here) 当前位置显示表格,
%如果实在不行显示在 b(bottom) 底部,

\begin{tabular}{|c|c|c|c|c|c|}   %开始绘制表格
%{|c|c|c|c|c|} 表示会有5列, 每个的方式未居中(c),
%也可以改成靠左(l)和靠右(r) 其中 | 表示绘制列线
\hline %绘制一条水平的线
\hline %再绘制一条水平的线
Time        & Monday       & Tuesday           & Wednesday  & Thursday      & Friday \\
\hline
07:00-08:00 &              &                   &            &               &  \\
\hline
08:00-09:00 &              &                   &            &               &  \\
\hline
\hline
09:00-10:30 &              & 9-10 Xin          &            &               & 930 210EP216  \\
            &              & 10-11 Xin Sanjeev &            &               &  \\
\hline
10:30-11:30 & EC JEB 025   & 11-11:30 Matthew  & EC JEB 02  & 11-12:15 NN26 & EC JEB 025 \\
\hline
11:30-12:30 & 541 JEB 21   & Matthew Keith     & 541 JEB 21 & 11-12:15 NN26 & 541 JEB 21 \\
            & 270 TLC148   & Matthew Keith     & 270 TLC148 &               & 270 TLC148 \\
\hline
12:30-1:00  & Xin Sanjeev  &                   &            &               & \\
\hline
1:00-2:00   & Xin          & Xin Matthew Keith & 1:30-2:30 324 121 & 1:30-2:30 580 & \\
\hline
2:00-2:30   & me           & Xin Matthew Keith & 1:30-2:30 324 121 & 1:30-2:30 580 & John   \\
2:30-3:00   & Andrew       & Matthew Sanjeev   & mainOffice & mainOffice    & Alex   \\
\hline
3:00-4:00   & Andrew       & Matthew Sanjeev   & mainOffice & mainOffice    & Alex   \\
\hline
4:00-4:30   & Andrew Keith & me                & mainOffice & mainOffice    & Alex Keith\\
4:30-5:00   & 499 JEB326   & me                & mainOffice & mainOffice    & Alex Keith\\
\hline
\hline
5:00-7:00   &  &  &  &  & \\
\hline
\hline
7:00-9:00   &  &  &  &  & \\
\hline
9:00-11:00  &  &  &  &  & \\
\hline
\hline

%下面这一段有点复杂,参加后面的解释,可以自己修改慢慢体会.
%\multirow{2}{*}{Multi-Row} & \multicolumn{2}{|c|}{Multi-Column} & %\multicolumn{2}{|c|}{\multirow{2}{*}{Multi-Row and Col}} \\
%上面开始两行合并, 然后又是正常的两列合并, 接下来是两行两列合并
%\cline{2-3} %绘制第2列和第3列的横线
%& column-1 & column-2 & \multicolumn{2}{|c|}{}\\
%补偿上面的两列合并的那一行
\end{tabular}
\end{table}


Week: \#\ \ \ \ \ \ \ \ \  Date: 0\ \ /\ \ \ \ /2014 
\begin{table}[!hbp]   %开始表格
%其中参数[!hbp] 的意思是:
%!表示尽可能的尝试 h(here) 当前位置显示表格,
%如果实在不行显示在 b(bottom) 底部,

\begin{tabular}{|c|c|c|c|c|c|}   %开始绘制表格
%{|c|c|c|c|c|} 表示会有5列, 每个的方式未居中(c),
%也可以改成靠左(l)和靠右(r) 其中 | 表示绘制列线
\hline %绘制一条水平的线
\hline %再绘制一条水平的线
Time        & Monday       & Tuesday           & Wednesday  & Thursday      & Friday \\
\hline
07:00-08:00 &              &                   &            &               &  \\
\hline
08:00-09:00 &              &                   &            &               &  \\
\hline
\hline
09:00-10:30 &              & 9-10 Xin          &            &               & 930 210EP216 \\
            &              & 10-11 Xin Sanjeev &            &               &  \\
\hline
10:30-11:30 & EC JEB 025   & 11-11:30 Matthew  & EC JEB 02  & 11-12:15 NN26 & EC JEB 025 \\
\hline
11:30-12:30 & 541 JEB 21   & Matthew Keith     & 541 JEB 21 & 11-12:15 NN26 & 541 JEB 21 \\
            & 270 TLC148   & Matthew Keith     & 270 TLC148 &               & 270 TLC148 \\
\hline
12:30-1:00  & Xin Sanjeev  &                   &            &               & \\
\hline
1:00-2:00   & Xin          & Xin Matthew Keith & 1:30-2:30 324 121 & 1:30-2:30 580 & \\
\hline
2:00-2:30   & me           & Xin Matthew Keith & 1:30-2:30 324 121 & 1:30-2:30 580 & John   \\
2:30-3:00   & Andrew       & Matthew Sanjeev   & mainOffice & mainOffice    & Alex   \\
\hline
3:00-4:00   & Andrew       & Matthew Sanjeev   & mainOffice & mainOffice    & Alex   \\
\hline
4:00-4:30   & Andrew Keith & me                & mainOffice & mainOffice    & Alex Keith\\
4:30-5:00   & 499 JEB326   & me                & mainOffice & mainOffice    & Alex Keith\\
\hline
\hline
5:00-7:00   &  &  &  &  & \\
\hline
\hline
7:00-9:00   &  &  &  &  & \\
\hline
9:00-11:00  &  &  &  &  & \\
\hline
\hline

%下面这一段有点复杂,参加后面的解释,可以自己修改慢慢体会.
%\multirow{2}{*}{Multi-Row} & \multicolumn{2}{|c|}{Multi-Column} & %\multicolumn{2}{|c|}{\multirow{2}{*}{Multi-Row and Col}} \\
%上面开始两行合并, 然后又是正常的两列合并, 接下来是两行两列合并
%\cline{2-3} %绘制第2列和第3列的横线
%& column-1 & column-2 & \multicolumn{2}{|c|}{}\\
%补偿上面的两列合并的那一行
\end{tabular}

%\caption{My Spring 2014 Class Schedule table} %表格的名称
\end{table}

\newpage
Week: \#\ \ \ \ \ \ \ \ \  Date: 0\ \ /\ \ \ \ /2014 \\
\begin{table}[!hbp]   %开始表格
%其中参数[!hbp] 的意思是:
%!表示尽可能的尝试 h(here) 当前位置显示表格,
%如果实在不行显示在 b(bottom) 底部,

\begin{tabular}{|c|c|c|c|c|c|}   %开始绘制表格
%{|c|c|c|c|c|} 表示会有5列, 每个的方式未居中(c),
%也可以改成靠左(l)和靠右(r) 其中 | 表示绘制列线
\hline %绘制一条水平的线
\hline %再绘制一条水平的线
Time        & Monday       & Tuesday           & Wednesday  & Thursday      & Friday \\
\hline
07:00-08:00 &              &                   &            &               &  \\
\hline
08:00-09:00 &              &                   &            &               &  \\
\hline
\hline
09:00-10:30 &              & 9-10 Xin          &            &               & 930 210EP216  \\
            &              & 10-11 Xin Sanjeev &            &               &  \\
\hline
10:30-11:30 & EC JEB 025   & 11-11:30 Matthew  & EC JEB 02  & 11-12:15 NN26 & EC JEB 025 \\
\hline
11:30-12:30 & 541 JEB 21   & Matthew Keith     & 541 JEB 21 & 11-12:15 NN26 & 541 JEB 21 \\
            & 270 TLC148   & Matthew Keith     & 270 TLC148 &               & 270 TLC148 \\
\hline
12:30-1:00  & Xin Sanjeev  &                   &            &               & \\
\hline
1:00-2:00   & Xin          & Xin Matthew Keith & 1:30-2:30 324 121 & 1:30-2:30 580 & \\
\hline
2:00-2:30   & me           & Xin Matthew Keith & 1:30-2:30 324 121 & 1:30-2:30 580 & John   \\
2:30-3:00   & Andrew       & Matthew Sanjeev   & mainOffice & mainOffice    & Alex   \\
\hline
3:00-4:00   & Andrew       & Matthew Sanjeev   & mainOffice & mainOffice    & Alex   \\
\hline
4:00-4:30   & Andrew Keith & me                & mainOffice & mainOffice    & Alex Keith\\
4:30-5:00   & 499 JEB326   & me                & mainOffice & mainOffice    & Alex Keith\\
\hline
\hline
5:00-7:00   &  &  &  &  & \\
\hline
\hline
7:00-9:00   &  &  &  &  & \\
\hline
9:00-11:00  &  &  &  &  & \\
\hline
\hline
%下面这一段有点复杂,参加后面的解释,可以自己修改慢慢体会.
%\multirow{2}{*}{Multi-Row} & \multicolumn{2}{|c|}{Multi-Column} & %\multicolumn{2}{|c|}{\multirow{2}{*}{Multi-Row and Col}} \\
%上面开始两行合并, 然后又是正常的两列合并, 接下来是两行两列合并
%\cline{2-3} %绘制第2列和第3列的横线
%& column-1 & column-2 & \multicolumn{2}{|c|}{}\\
%补偿上面的两列合并的那一行
\end{tabular}
\end{table}


Week: \#\ \ \ \ \ \ \ \ \  Date: 0\ \ /\ \ \ \ /2014 \\
\begin{table}[!hbp]   %开始表格
%其中参数[!hbp] 的意思是:
%!表示尽可能的尝试 h(here) 当前位置显示表格,
%如果实在不行显示在 b(bottom) 底部,

\begin{tabular}{|c|c|c|c|c|c|}   %开始绘制表格
%{|c|c|c|c|c|} 表示会有5列, 每个的方式未居中(c),
%也可以改成靠左(l)和靠右(r) 其中 | 表示绘制列线
\hline %绘制一条水平的线
\hline %再绘制一条水平的线
Time        & Monday       & Tuesday           & Wednesday  & Thursday      & Friday \\
\hline
07:00-08:00 &              &                   &            &               &  \\
\hline
08:00-09:00 &              &                   &            &               &  \\
\hline
\hline
09:00-10:30 &              & 9-10 Xin          &            &               & 930 210EP216  \\
            &              & 10-11 Xin Sanjeev &            &               &  \\
\hline
10:30-11:30 & EC JEB 025   & 11-11:30 Matthew  & EC JEB 02  & 11-12:15 NN26 & EC JEB 025 \\
\hline
11:30-12:30 & 541 JEB 21   & Matthew Keith     & 541 JEB 21 & 11-12:15 NN26 & 541 JEB 21 \\
            & 270 TLC148   & Matthew Keith     & 270 TLC148 &               & 270 TLC148 \\
\hline
12:30-1:00  & Xin Sanjeev  &                   &            &               & \\
\hline
1:00-2:00   & Xin          & Xin Matthew Keith & 1:30-2:30 324 121 & 1:30-2:30 580 & \\
\hline
2:00-2:30   & me           & Xin Matthew Keith & 1:30-2:30 324 121 & 1:30-2:30 580 & John   \\
2:30-3:00   & Andrew       & Matthew Sanjeev   & mainOffice & mainOffice    & Alex   \\
\hline
3:00-4:00   & Andrew       & Matthew Sanjeev   & mainOffice & mainOffice    & Alex   \\
\hline
4:00-4:30   & Andrew Keith & me                & mainOffice & mainOffice    & Alex Keith\\
4:30-5:00   & 499 JEB326   & me                & mainOffice & mainOffice    & Alex Keith\\
\hline
\hline
5:00-7:00   &  &  &  &  & \\
\hline
\hline
7:00-9:00   &  &  &  &  & \\
\hline
9:00-11:00  &  &  &  &  & \\
\hline
\hline

%下面这一段有点复杂,参加后面的解释,可以自己修改慢慢体会.
%\multirow{2}{*}{Multi-Row} & \multicolumn{2}{|c|}{Multi-Column} & %\multicolumn{2}{|c|}{\multirow{2}{*}{Multi-Row and Col}} \\
%上面开始两行合并, 然后又是正常的两列合并, 接下来是两行两列合并
%\cline{2-3} %绘制第2列和第3列的横线
%& column-1 & column-2 & \multicolumn{2}{|c|}{}\\
%补偿上面的两列合并的那一行
\end{tabular}
\end{table}

\end{document}