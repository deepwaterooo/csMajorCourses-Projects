% Created 2014-02-25 二 21:31
\documentclass[10pt]{article}
\usepackage[utf8]{inputenc}
\usepackage[T1]{fontenc}
\usepackage{fixltx2e}
\usepackage{graphicx}
\usepackage{longtable}
\usepackage{float}
\usepackage{wrapfig}
\usepackage{rotating}
\usepackage[normalem]{ulem}
\usepackage{amsmath}
\usepackage{textcomp}
\usepackage{marvosym}
\usepackage{wasysym}
\usepackage{amssymb}
\usepackage{hyperref}
\tolerance=1000
\usepackage{CJKutf8}
\usepackage{listings}
\usepackage{geometry}
\geometry{left=0cm,right=0cm,top=0cm,bottom=0cm}

\begin{CJK}{UTF8}{gbsn}
\author{jenny}
\date{\today}
\title{Lisp的本质}
\hypersetup{
  pdfkeywords={},
  pdfsubject={},
  pdfcreator={Emacs 24.3.50.1 (Org mode 8.2.5h)}}
\begin{document}

\maketitle
\tableofcontents



\lstset{language=c++,
numbers=left, 
numberstyle=\tiny,
basicstyle=\ttfamily\small,
tabsize=4,
frame=none, 
escapeinside=``, 
extendedchars=false
}


Lisp的本质(The Nature of Lisp)

作者 Slava Akhmechet
译者 Alec Jang

出处: \url{http://www.defmacro.org/ramblings/lisp.html}


\section{简介}
\label{sec-1}

最初在web的某些角落偶然看到有人赞美Lisp时, 我那时已经是一个颇有经验的程序员。在我的履历上, 掌握的语言范围相当广泛, 象C++, Java, C\#主流语言等等都不在话下,我觉得我差不多知道所有的有关编程语言的事情。对待编程语言的问题上, 我觉得自己不太会遇到什么大问题。其实我大错特错了。

我试着学了一下Lisp, 结果马上就撞了墙。我被那些范例代码吓坏了。我想很多初次接触Lisp语言的人, 一定也有过类似的感受。Lisp的语法太次了。一个语言的发明人, 居然不肯用心弄出一套漂亮的语法, 那谁还会愿意学它。反正, 我是确确实实被那些难看的无数的括号搞蒙了。

回过神来之后, 我和Lisp社区的那伙人交谈, 诉说我的沮丧心情。结果, 立马就有一大套理论砸过来, 这套理论在Lisp社区处处可见, 几成惯例。比如说: Lisp的括号只是表面现象; Lisp的代码和数据的表达方式没有差别, 而且比XML语法高明许多, 所以有无穷的好处; Lisp有强大无比的元语言能力, 程序员可以写出自我维护的代码; Lisp可以创造出针对特定应用的语言子集; Lisp的运行时和编译时没有明确的分界; 等等, 等等, 等等。这么长的赞美词虽然看起来相当动人, 不过对我毫无意义。没人能给我演示这些东西是如何应用的, 因为这些东西一般来说只有在大型系统才会用到。我争辩说, 这些东西传统语言一样办得到。在和别人争论了数个小时之后, 我最终还是放弃了学Lisp的念头。为什么要花费几个月的时间学习语法这么难看的语言呢? 这种语言的概念这么晦涩, 又没什么好懂的例子。也许这语言不是该我这样的人学的。

几个月来, 我承受着这些Lisp辩护士对我心灵的重压。我一度陷入了困惑。我认识一些绝顶聪明的人, 我对他们相当尊敬, 我看到他们对Lisp的赞美达到了宗教般的高度。这就是说, Lisp中一定有某种神秘的东西存在, 我不能忍受自己对此的无知, 好奇心和求知欲最终不可遏制。我于是咬紧牙关埋头学习Lisp, 经过几个月的时间费劲心力的练习, 终于,我看到了那无穷无尽的泉水的源头。在经过脱胎换骨的磨练之后, 在经过七重地狱的煎熬之后, 终于, 我明白了。

顿悟在突然之间来临。曾经许多次, 我听到别人引用雷蒙德(译者注: 论文\label{}的作者, 著名的黑客社区理论家)的话: "Lisp语言值得学习。当你学会Lisp之后, 你会拥有深刻的体验。就算你平常并不用Lisp编程, 它也会使你成为更加优秀的程序员"。过去, 我根本不懂这些话的含义, 我也不相信这是真的。可是现在我懂得了。这些话蕴含的真理远远超过我过去的想像。我内心体会到一种神圣的情感, 一瞬间的顿悟, 几乎使我对电脑科学的观念发生了根本的改变。

顿悟的那一刻, 我成了Lisp的崇拜者。我体验到了宗教大师的感受: 一定要把我的知识传布开来, 至少要让10个迷失的灵魂得到拯救。按照通常的办法, 我把这些道理(就是刚开始别人砸过来的那一套, 不过现在我明白了真实的含义)告诉旁人。结果太令人失望了, 只有少数几个人在我坚持之下, 发生了一点兴趣, 但是仅仅看了几眼Lisp代码, 他们就退却了。照这样的办法, 也许费数年功夫能造就了几个Lisp迷, 但我觉得这样的结果太差强人意了, 我得想一套有更好的办法。

我深入地思考了这个问题。是不是Lisp有什么很艰深的东西, 令得那么多老练的程序员都不能领会? 不是, 没有任何绝对艰深的东西。因为我能弄懂, 我相信其他人也一定能。那么问题出在那里? 后来我终于找到了答案。我的结论就是, 凡是教人学高级概念, 一定要从他已经懂得的东西开始。如果学习过程很有趣, 学习的内容表达得很恰当, 新概念就会变得相当直观。这就是我的答案。所谓元编程, 所谓数据和代码形式合一, 所谓自修改代码, 所谓特定应用的子语言, 所有这些概念根本就是同族概念, 彼此互为解释, 肯定越讲越不明白。还是从实际的例子出发最有用。

我把我的想法说给Lisp程序员听, 遭到了他们的反对。"这些东西本身当然不可能用熟悉的知识来解释, 这些概念完全与众不同, 你不可能在别人已有的经验里找到类似的东西",可是我认为这些都是遁词。他们又反问我, "你自己为啥不试一下?" 好吧, 我来试一下。这篇文章就是我尝试的结果。我要用熟悉的直观的方法来解释Lisp, 我希望有勇气的人读完它, 拿杯饮料, 深呼吸一下, 准备被搞得晕头转向。来吧, 愿你获得大能。
\section{重新审视XML}
\label{sec-2}

千里之行始于足下。让我们的第一步从XML开始。可是XML已经说得更多的了, 还能有什么新意思可说呢? 有的。XML自身虽然谈谈不上有趣, 但是XML和Lisp的关系却相当有趣。XML和Lisp的概念有着惊人的相似之处。XML是我们通向理解Lisp的桥梁。好吧, 我们且把XML当作活马医。让我们拿好手杖, 对XML的无人涉及的荒原地带作一番探险。我们要从一个全新的视角来考察这个题目。

表面上看, XML是一种标准化语法, 它以适合人阅读的格式来表达任意的层次化数据(hirearchical data)。象任务表(to-do list), 网页, 病历, 汽车保险单, 配置文件等等, 都是XML用武的地方。比如我们拿任务表做例子:
\begin{lstlisting}[language=xml]
<todo name="housework">
    <item priority="high">Clean the house.</item>
    <item priority="medium">Wash the dishes.</item>
    <item priority="medium">Buy more soap.</item>
</todo>
\end{lstlisting}

解析这段数据时会发生什么情况? 解析之后的数据在内存中怎样表示? 显然, 用树来表示这种层次化数据是很恰当的。说到底, XML这种比较容易阅读的数据格式, 就是树型结构数据经过序列化之后的结果。任何可以用树来表示的数据, 同样可以用XML来表示, 反之亦然。希望你能懂得这一点, 这对下面的内容极其重要。

再进一步。还有什么类型的数据也常用树来表示? 无疑列表(list)也是一种。上过编译课吧? 还模模糊糊记得一点吧? 源代码在解析之后也是用树结构来存放的, 任何编译程序都会把源代码解析成一棵抽象语法树, 这样的表示法很恰当, 因为源代码就是层次结构的:函数包含参数和代码块, 代码快包含表达式和语句, 语句包含变量和运算符等等。

我们已经知道, 任何树结构都可以轻而易举的写成XML, 而任何代码都会解析成树, 因此,任何代码都可以转换成XML, 对不对? 我举个例子, 请看下面的函数:
\begin{lstlisting}[language=c++]
int add(int arg1, int arg2)
{
    return arg1+arg2;
}
\end{lstlisting}

能把这个函数变成对等的XML格式吗? 当然可以。我们可以用很多种方式做到, 下面是其中的一种, 十分简单:
\begin{lstlisting}[language=xml]
<define-function return-type="int" name="add">
    <arguments>
        <argument type="int">arg1</argument>
        <argument type="int">arg2</argument>
    </arguments>
    <body>
        <return>
            <add value1="arg1" value2="arg2" />
        </return>
    </body>
</define>
\end{lstlisting}

这个例子非常简单, 用哪种语言来做都不会有太大问题。我们可以把任何程序码转成XML,也可以把XML转回到原来的程序码。我们可以写一个转换器, 把Java代码转成XML, 另一个转换器把XML转回到Java。一样的道理, 这种手段也可以用来对付C++(这样做跟发疯差不多么。可是的确有人在做, 看看GCC-XML(\url{http://www.gccxml.org})就知道了)。进一步说,凡是有相同语言特性而语法不同的语言, 都可以把XML当作中介来互相转换代码。实际上几乎所有的主流语言都在一定程度上满足这个条件。我们可以把XML作为一种中间表示法,在两种语言之间互相译码。比方说, 我们可以用Java2XML把Java代码转换成XML, 然后用XML2CPP再把XML转换成C++代码, 运气好的话, 就是说, 如果我们小心避免使用那些C++不具备的Java特性的话, 我们可以得到完好的C++程序。这办法怎么样, 漂亮吧?

这一切充分说明, 我们可以把XML作为源代码的通用存储方式, 其实我们能够产生一整套使用统一语法的程序语言, 也能写出转换器, 把已有代码转换成XML格式。如果真的采纳这种办法, 各种语言的编译器就用不着自己写语法解析了, 它们可以直接用XML的语法解析来直接生成抽象语法树。

说到这里你该问了, 我们研究了这半天XML, 这和Lisp有什么关系呢? 毕竟XML出来之时,Lisp早已经问世三十年了。这里我可以保证, 你马上就会明白。不过在继续解释之前, 我们先做一个小小的思维练习。看一下上面这个XML版本的add函数例子, 你怎样给它分类,是代码还是数据? 不用太多考虑都能明白, 把它分到哪一类都讲得通。它是XML, 它是标准格式的数据。我们也知道, 它可以通过内存中的树结构来生成(GCC-XML做的就是这个事情)。它保存在不可执行的文件中。我们可以把它解析成树节点, 然后做任意的转换。显而易见, 它是数据。不过且慢, 虽然它语法有点陌生, 可它又确确实实是一个add函数, 对吧?  一旦经过解析, 它就可以拿给编译器编译执行。我们可以轻而易举写出这个XML 代码解释器, 并且直接运行它。或者我们也可以把它译成Java或C++代码, 然后再编译运行。所以说, 它也是代码。

我们说到那里了? 不错, 我们已经发现了一个有趣的关键之点。过去被认为很难解的概念已经非常直观非常简单的显现出来。代码也是数据, 并且从来都是如此。这听起来疯疯癫癫的, 实际上却是必然之事。我许诺过会以一种全新的方式来解释Lisp, 我要重申我的许诺。但是我们此刻还没有到预定的地方, 所以还是先继续上边的讨论。

刚才我说过, 我们可以非常简单地实现XML版的add函数解释器, 这听起来好像不过是说说而已。谁真的会动手做一下呢? 未必有多少人会认真对待这件事。随便说说, 并不打算真的去做, 这样的事情你在生活中恐怕也遇到吧。你明白我这样说的意思吧, 我说的有没有打动你? 有哇, 那好, 我们继续。

重新审视Ant

我们现在已经来到了月亮背光的那一面, 先别忙着离开。再探索一下, 看看我们还能发现什么东西。闭上眼睛, 想一想2000年冬天的那个雨夜, 一个名叫James Duncan Davidson的杰出的程序员正在研究Tomcat的servlet容器。那时, 他正小心地保存好刚修改过的文件, 然后执行make。结果冒出了一大堆错误, 显然有什么东西搞错了。经过仔细检查, 他想, 难道是因为tab前面加了个空格而导致命令不能执行吗? 确实如此。老是这样, 他真的受够了。乌云背后的月亮给了他启示, 他创建了一个新的Java项目, 然后写了一个简单但是十分有用的工具, 这个工具巧妙地利用了Java属性文件中的信息来构造工程, 现在James可以写makefile的替代品, 它能起到相同的作用, 而形式更加优美, 也不用担心有makefile那样可恨的空格问题。这个工具能够自动解释属性文件, 然后采取正确的动作来编译工程。真是简单而优美。

(作者注: 我不认识James, James也不认识我, 这个故事是根据网上关于Ant历史的帖子虚构的)

使用Ant构造Tomcat之后几个月, 他越来越感到Java的属性文件不足以表达复杂的构造指令。文件需要检出, 拷贝, 编译, 发到另外一台机器, 进行单元测试。要是出错, 就发邮件给相关人员, 要是成功, 就继续在尽可能高层的卷(volumn)上执行构造。追踪到最后,卷要回复到最初的水平上。确实, Java的属性文件不够用了, James需要更有弹性的解决方案。他不想自己写解析器(因为他更希望有一个具有工业标准的方案)。XML看起来是个不错的选择。他花了几天工夫把Ant移植到XML,于是,一件伟大的工具诞生了。

Ant是怎样工作的?原理非常简单。Ant把包含有构造命令的XML文件(算代码还是算数据,你自己想吧),交给一个Java程序来解析每一个元素,实际情况比我说的还要简单得多。一个简单的XML指令会导致具有相同名字的Java类装入,并执行其代码。

<copy todir="../new/dir">
    <fileset dir="src$_{\text{dir}}$" />
</copy>

这段文字的含义是把源目录复制到目标目录,Ant会找到一个"copy"任务(实际上就是一个Java类), 通过调用Java的方法来设置适当参数(todir和fileset),然后执行这个任务。Ant带有一组核心类, 可以由用户任意扩展, 只要遵守若干约定就可以。Ant找到这些类,每当遇到XML元素有同样的名字, 就执行相应的代码。过程非常简单。Ant做到了我们前面所说的东西: 它是一个语言解释器, 以XML作为语法, 把XML元素转译为适当的Java指令。我们可以写一个"add"任务, 然后, 当发现XML中有add描述的时候, 就执行这个add任务。由于Ant是非常流行的项目, 前面展示的策略就显得更为明智。毕竟, 这个工具每天差不多有几千家公司在使用。

到目前为之, 我还没有说Ant在解析XML时所遇到困难。你也不用麻烦去它的网站上去找答案了, 不会找到有价值的东西。至少对我们这个论题来说是如此。我们还是继续下一步讨论吧。我们答案就在那里。
\section{为什么是XML}
\label{sec-3}

有时候正确的决策并非完全出于深思熟虑。我不知道James选择XML是否出于深思熟虑。也许仅仅是个下意识的决定。至少从James在Ant网站上发表的文章看起来, 他所说的理由完全是似是而非。他的主要理由是移植性和扩展性, 在Ant案例上, 我看不出这两条有什么帮助。使用XML而不是Java代码, 到底有什么好处? 为什么不写一组Java类, 提供api来满足基本任务(拷贝目录, 编译等等), 然后在Java里直接调用这些代码? 这样做仍然可以保证移植性, 扩展性也是毫无疑问的。而且语法也更为熟悉, 看着顺眼。那为什么要用 XML呢? 有什么更好的理由吗?

有的。虽然我不确定James是否确实意识到了。在语义的可构造性方面, XML的弹性是Java望尘莫及的。我不想用高深莫测的名词来吓唬你, 其中的道理相当简单, 解释起来并不费很多功夫。好, 做好预备动作, 我们马上就要朝向顿悟的时刻做奋力一跃。

上面的那个copy的例子, 用Java代码怎样实现呢? 我们可以这样做:
\begin{lstlisting}[language=java]
CopyTask copy = new CopyTask();
Fileset fileset = new Fileset();

fileset.setDir("src_dir");
copy.setToDir("../new/dir");
copy.setFileset(fileset);

copy.excute();
\end{lstlisting}

这个代码看起来和XML的那个很相似, 只是稍微长一点。差别在那里? 差别在于XML构造了一个特殊的copy动词, 如果我们硬要用Java来写的话, 应该是这个样子:
\begin{lstlisting}[language=java]
    copy("../new/dir");
    {
        fileset("src_dir");
    }
\end{lstlisting}

看到差别了吗? 以上代码(如果可以在Java中用的化), 是一个特殊的copy算符, 有点像for循环或者Java5中的foreach循环。如果我们有一个转换器, 可以把XML转换到Java, 大概就会得到上面这段事实上不可以执行的代码。因为Java的技术规范是定死的, 我们没有办法在程序里改变它。我们可以增加包, 增加类, 增加方法, 但是我们没办法增加算符, 而对于XML, 我们显然可以任由自己增加这样的东西。对于XML的语法树来说, 只要原意,我们可以任意增加任何元素, 因此等于我们可以任意增加算符。如果你还不太明白的话, 看下面这个例子, 加入我们要给Java引入一个unless算符:
\begin{lstlisting}[language=java]
    unless(someObject.canFly())
    {
        someObject.transportByGround():
    }
\end{lstlisting}

在上面的两个例子中, 我们打算给Java语法扩展两个算符, 成组拷贝文件算符和条件算符unless, 我们要想做到这一点, 就必须修改Java编译器能够接受的抽象语法树, 显然我们无法用Java标准的功能来实现它。但是在XML中我们可以轻而易举地做到。我们的解析器根据 XML元素, 生成抽象语法树, 由此生成算符, 所以, 我们可以任意引入任何算符。

对于复杂的算符来说, 这样做的好处显而易见。比如, 用特定的算符来做检出源码, 编译文件, 单元测试, 发送邮件等任务, 想想看有多么美妙。对于特定的题目, 比如说构造软件项目, 这些算符的使用可以大幅减低少代码的数量。增加代码的清晰程度和可重用性。解释性的XML可以很容易的达到这个目标。XML是存储层次化数据的简单数据文件, 而在Java中, 由于层次结构是定死的(你很快就会看到, Lisp的情况与此截然不同), 我们就没法达到上述目标。也许这正是Ant的成功之处呢。

你可以注意一下最近Java和C\#的变化(尤其是C\#3.0的技术规范), C\#把常用的功能抽象出来, 作为算符增加到C\#中。C\#新增加的query算符就是一个例子。它用的还是传统的作法:C\#的设计者修改抽象语法树, 然后增加对应的实现。如果程序员自己也能修改抽象语法树该有多好! 那样我们就可以构造用于特定问题的子语言(比如说就像Ant这种用于构造项目的语言), 你能想到别的例子吗? 再思考一下这个概念。不过也不必思考太甚, 我们待会还会回到这个题目。那时候就会更加清晰。
\section{离Lisp越来越近}
\label{sec-4}

我们先把算符的事情放一放, 考虑一下Ant设计局限之外的东西。我早先说过, Ant可以通过写Java类来扩展。Ant解析器会根据名字来匹配XML元素和Java类, 一旦找到匹配, 就执行相应任务。为什么不用Ant自己来扩展Ant呢? 毕竟核心任务要包含很多传统语言的结构(例如"if"), 如果Ant自身就能提供构造任务的能力(而不是依赖java类), 我们就可以得到更高的移植性。我们将会依赖一组核心任务(如果你原意, 也不妨把它称作标准库), 而不用管有没有Java 环境了。这组核心任务可以用任何方式来实现, 而其他任务建筑在这组核心任务之上, 那样的话, Ant就会成为通用的, 可扩展的, 基于XML的编程语言。考虑下面这种代码的可能性:
\begin{lstlisting}[language=xml]
  <task name="Test">
        <echo message="Hello World" />
    </task>
    <Test />
\end{lstlisting}

如果XML支持"task"的创建, 上面这段代码就会输出"Hello World!". 实际上, 我们可以用Java写个"task"任务, 然后用Ant-XML来扩展它。Ant可以在简单原语的基础上写出更复杂的原语, 就像其他编程语言常用的作法一样。这也就是我们一开始提到的基于XML的编程语言。这样做用处不大(你知道为甚么吗?), 但是真的很酷。

再看一回我们刚才说的Task任务。祝贺你呀, 你在看Lisp代码!!! 我说什么? 一点都不像Lisp吗? 没关系, 我们再给它收拾一下。
\section{比XML更好}
\label{sec-5}

前面一节说过, Ant自我扩展没什么大用, 原因在于XML很烦琐。对于数据来说, 这个问题还不太大, 但如果代码很烦琐的话, 光是打字上的麻烦就足以抵消它的好处。你写过Ant 的脚本吗? 我写过, 当脚本达到一定复杂度的时候, XML非常让人厌烦。想想看吧, 为了写结束标签, 每个词都得打两遍, 不发疯算好的!

为了解决这个问题, 我们应当简化写法。须知, XML仅仅是一种表达层次化数据的方式。我们并不是一定要使用尖括号才能得到树的序列化结果。我们完全可以采用其他的格式。其中的一种(刚好就是Lisp所采用的)格式, 叫做s表达式。s表达式要做的和XML一样, 但它的好处是写法更简单, 简单的写法更适合代码输入。后面我会详细讲s表达式。这之前我要清理一下XML的东西。考虑一下关于拷贝文件的例子:
\begin{lstlisting}[language=xml]
    <copy toDir="../new/dir">
        <fileset dir="src_dir">
    </copy>
\end{lstlisting}

想想看在内存里面, 这段代码的解析树在内存会是什么样子? 会有一个"copy"节点, 其下有一个 "fileset"节点, 但是属性在哪里呢? 它怎样表达呢? 如果你以前用过XML, 并且弄不清楚该用元素还是该用属性, 你不用感到孤单, 别人一样糊涂着呢。没人真的搞得清楚。这个选择与其说是基于技术的理由, 还不如说是闭着眼瞎摸。从概念上来讲, 属性也是一种元素, 任何属性能做的, 元素一样做得到。XML引入属性的理由, 其实就是为了让XML写法不那么冗长。比如我们看个例子:
\begin{lstlisting}[language=xml]
    <copy>
        <toDir>../new/dir</toDir>
        <fileset>
            <dir>src_dir</dir>
        </fileset>
    </copy>
\end{lstlisting}

两下比较, 内容的信息量完全一样, 用属性可以减少打字数量。如果XML没有属性的话, 光是打字就够把人搞疯掉。

说完了属性的问题, 我们再来看一看s表达式。之所以绕这么个弯, 是因为s表达式没有属性的概念。因为s表达式非常简练, 根本没有必要引入属性。我们在把XML转换成s表达式的时候, 心里应该记住这一点。看个例子, 上面的代码译成s表达式是这样的:
\begin{lstlisting}[language=lisp]
    (copy 
        (todir "../new/dir")
        (fileset (dir "src_dir")))
\end{lstlisting}

仔细看看这个例子, 差别在哪里? 尖括号改成了圆括号, 每个元素原来是有一对括号标记包围的, 现在取消了后一个(就是带斜杠的那个)括号标记。表示元素的结束只需要一个")"就可以了。不错, 差别就是这些。这两种表达方式的转换, 非常自然, 也非常简单。s表达式打起字来, 也省事得多。第一次看s表达式(Lisp)时, 括号很烦人是吧? 现在我们明白了背后的道理, 一下子就变得容易多了。至少, 比XML要好的多。用s表达式写代码, 不单是实用, 而且也很让人愉快。s表达式具有XML的一切好处, 这些好处是我们刚刚探讨过的。现在我们看看更加Lisp风格的task例子:
\begin{lstlisting}[language=lisp]
    (task (name "Test")
        (echo (message "Hellow World!")))
    (Test)
\end{lstlisting}

用Lisp的行话来讲, s表达式称为表(list)。对于上面的例子, 如果我们写的时候不加换行, 用逗号来代替空格, 那么这个表达式看起来就非常像一个元素列表, 其中又嵌套着其他标记。
\begin{lstlisting}[language=lisp]
    (task, (name, "test"), (echo, (message, "Hello World!")))
\end{lstlisting}

XML自然也可以用这样的风格来写。当然上面这句并不是一般意义上的元素表。它实际上是一个树。这和XML的作用是一样的。称它为列表, 希望你不会感到迷惑, 因为嵌套表和树实际上是一码事。Lisp的字面意思就是表处理(list processing), 其实也可以称为树处理, 这和处理XML节点没有什么不同。

经受这一番折磨以后, 现在我们终于相当接近Lisp了, Lisp的括号的神秘本质(就像许多Lisp狂热分子认为的)逐渐显现出来。现在我们继续研究其他内容。
\section{重新审视C语言的宏}
\label{sec-6}

到了这里, 对XML的讨论你大概都听累了, 我都讲累了。我们先停一停, 把树, s表达式,Ant这些东西先放一放, 我们来说说C的预处理器。一定有人问了, 我们的话题和C有什么关系? 我们已经知道了很多关于元编程的事情, 也探讨过专门写代码的代码。理解这问题有一定难度, 因为相关讨论文章所使用的编程语言, 都是你们不熟悉的。但是如果只论概念的话, 就相对要简单一些。我相信, 如果以C语言做例子来讨论元编程, 理解起来一定会容易得多。好, 我们接着看。

一个问题是, 为什么要用代码来写代码呢? 在实际的编程中, 怎样做到这一点呢? 到底元编程是什么意思? 你大概已经听说过这些问题的答案, 但是并不懂得其中缘由。为了揭示背后的真理, 我们来看一下一个简单的数据库查询问题。这种题目我们都做过。比方说, 直接在程序码里到处写SQL语句来修改表(table)里的数据, 写多了就非常烦人。即便用C\#3.0的LINQ, 仍然不减其痛苦。写一个完整的SQL查询(尽管语法很优美)来修改某人的地址, 或者查找某人的名字, 绝对是件令程序员倍感乏味的事情, 那么我们该怎样来解决这个问题? 答案就是: 使用数据访问层。 

概念挺简单, 其要点是把数据访问的内容(至少是那些比较琐碎的部分)抽象出来, 用类来映射数据库的表, 然后用访问对象属性访问器(accessor)的办法来间接实现查询。这样就极大地简化了开发工作量。我们用访问对象的方法(或者属性赋值, 这要视你选用的语言而定)来代替写SQL查询语句。凡是用过这种方法的人, 都知道这很节省时间。当然, 如果你要亲自写这样一个抽象层, 那可是要花非常多的时间的--你要写一组类来映射表, 把属性访问转换为SQL查询, 这个活相当耗费精力。用手工来做显然是很不明智的。但是一旦你有了方案和模板, 实际上就没有多少东西需要思考的。你只需要按照同样的模板一次又一次重复编写相似代码就可以了。事实上很多人已经发现了更好的方法, 有一些工具可以帮助你连接数据库, 抓取数据库结构定义(schema), 按照预定义的或者用户定制的模板来自动编写代码。

如果你用过这种工具, 你肯定会对它的神奇效果深为折服。往往只需要鼠标点击数次, 就可以连接到数据库, 产生数据访问源码, 然后把文件加入到你的工程里面, 十几分钟的工作, 按照往常手工方式来作的话, 也许需要数百个小时人工(man-hours)才能完成。可是,如果你的数据库结构定义后来改变了怎么办? 那样的话, 你只需把这个过程重复一遍就可以了。甚至有一些工具能自动完成这项变动工作。你只要把它作为工程构造的一部分, 每次编译工程的时候, 数据库部分也会自动地重新构造。这真的太棒了。你要做的事情基本上减到了0。如果数据库结构定义发生了改变, 并在编译时自动更新了数据访问层的代码,那么程序中任何使用过时的旧代码的地方, 都会引发编译错误。

数据访问层是个很好的例子, 这样的例子还有好多。从GUI样板代码, WEB代码, COM和CORBA存根, 以及MFC和ATL等等。在这些地方, 都是有好多相似代码多次重复。既然这些代码有可能自动编写, 而程序员时间又远远比CPU时间昂贵, 当然就产生了好多工具来自动生成样板代码。这些工具的本质是什么呢? 它们实际上就是制造程序的程序。它们有一个神秘的名字, 叫做元编程。所谓元编程的本义, 就是如此。

元编程本来可以用到无数多的地方, 但实际上使用的次数却没有那么多。归根结底, 我们心里还是在盘算, 假设重复代码用拷贝粘贴的话, 大概要重复6,7次, 对于这样的工作量,值得专门建立一套生成工具吗? 当然不值得。数据访问层和COM存根往往需要重用数百次,甚至上千次, 所以用工具生成是最好的办法。而那些仅仅是重复几次十几次的代码, 是没有必要专门做工具的。不必要的时候也去开发代码生成工具, 那就显然过度估计了代码生成的好处。当然, 如果创建这类工具足够简单的话, 还是应当尽量多用, 因为这样做必然会节省时间。现在来看一下有没有合理的办法来达到这个目的。

现在, C预处理器要派上用场了。我们都用过C/C++的预处理器, 我们用它执行简单的编译指令, 来产生简单的代码变换(比方说, 设置调试代码开关), 看一个例子:
\begin{lstlisting}[language=c]
    #define triple(X) X+X+X
\end{lstlisting}

这一行的作用是什么? 这是一个简单的预编译指令, 它把程序中的triple(X)替换称为X+X+X。例如, 把所有的triple(5)都换成5+5+5, 然后再交给编译器编译。这就是一个简单的代码生成的例子。要是C的预处理器再强大一点, 要是能够允许连接数据库, 要是能多一些其他简单的机制, 我们就可以在我们程序的内部开发自己的数据访问层。下面这个例子, 是一个假想的对C宏的扩展:
\begin{lstlisting}[language=c]
    #get -db-schema("127.0.0.1")
    #iterate -through-tables
    #for -each-table
        class #table -name
            {
            };
    #end -for-each
\end{lstlisting}

我们连接数据库结构定义, 遍历数据表, 然后对每个表创建一个类, 只消几行代码就完成了这个工作。这样每次编译工程的时候, 这些类都会根据数据库的定义同步更新。显而易见, 我们不费吹灰之力就在程序内部建立了一个完整的数据访问层, 根本用不着任何外部工具。当然这种作法有一个缺点, 那就是我们得学习一套新的"编译时语言", 另一个缺点就是根本不存在这么一个高级版的C预处理器。需要做复杂代码生成的时候, 这个语言(译者注: 这里指预处理指令, 即作者所说的"编译时语言")本身也一定会变得相当复杂。它必须支持足够多的库和语言结构。比如说我们想要生成的代码要依赖某些ftp服务器上的文件, 预处理器就得支持ftp访问, 仅仅因为这个任务而不得不创造和学习一门新的语言,真是有点让人恶心(事实上已经存在着有此能力的语言, 这样做就更显荒谬)。我们不妨再灵活一点, 为什么不直接用 C/C++自己作为自己的预处理语言呢?  这样子的话, 我们可以发挥语言的强大能力, 要学的新东西也只不过是几个简单的指示字 , 这些指示字用来区别编译时代码和运行时代码。
\begin{lstlisting}[language=c]
    <%
        cout<<"Enter a number: ";
        cin>>n;
    %>
    for(int i=0;i< <% n %>;i++)
    {
        cout<<"hello"<<endl;
    }
\end{lstlisting}

你明白了吗? 在<\%和\%>标记之间的代码是在编译时运行的, 标记之外的其他代码都是普通代码。编译程序时, 系统会提示你输入一个数, 这个数在后面的循环中会用到。而for循环的代码会被编译。假定你在编译时输入5, for循环的代码将会是:
\begin{lstlisting}[language=c++]
    for(int i=0;i<5; i++)
    {
        cout<<"hello"<<endl;
    }
\end{lstlisting}

又简单又有效率, 也不需要另外的预处理语言。我们可以在编译时就充分发挥宿主语言( 此处是C/C++)的强大能力, 我们可以很容易地在编译时连接数据库, 建立数据访问层, 就像JSP或者ASP创建网页那样。我们也用不着专门的窗口工具来另外建立工程。我们可以在代码中立即加入必要的工具。我们也用不着顾虑建立这种工具是不是值得, 因为这太容易了, 太简单了。这样子不知可以节省多少时间啊。
\section{你好, Lisp}
\label{sec-7}

到此刻为止, 我们所知的关于Lisp的指示可以总结为一句话: Lisp是一个可执行的语法更优美的XML, 但我们还没有说Lisp是怎样做到这一点的, 现在开始补上这个话题。 

Lisp有丰富的内置数据类型, 其中的整数和字符串和其他语言没什么分别。像71或者"hello"这样的值, 含义也和C++或者Java这样的语言大体相同。真正有意思的三种类型是符号(symbol), 表和函数。这一章的剩余部分, 我都会用来介绍这几种类型, 还要介绍Lisp环境是怎样编译和运行源码的。这个过程用Lisp的术语来说通常叫做求值。通读这一节内容, 对于透彻理解元编程的真正潜力, 以及代码和数据的同一性, 和面向领域语言的观念, 都极其重要。万勿等闲视之。我会尽量讲得生动有趣一些, 也希望你能获得一些启发。那好, 我们先讲符号。

大体上, 符号相当于C++或Java语言中的标志符, 它的名字可以用来访问变量值(例如currentTime, arrayCount, n, 等等), 差别在于, Lisp中的符号更加基本。在C++或Java里面, 变量名只能用字母和下划线的组合, 而Lisp的符号则非常有包容性, 比如, 加号(+)就是一个合法的符号, 其他的像-, =, hello-world, *等等都可以是符号名。符号名的命名规则可以在网上查到。你可以给这些符号任意赋值, 我们这里先用伪码来说明这一点。假定函数set是给变量赋值(就像等号=在C++和Java里的作用), 下面是我们的例子:
\begin{lstlisting}[language=lisp]
    set(test, 5)            // 符号test的值为5
    set(=, 5)               // 符号=的值为5
    set(test, "hello")      // 符号test的值为字符串"hello"
    set(test, =)            // 此时符号=的值为5, 所以test的也为5
    set(* , "hello")         // 符号 * 的值为"hello"
\end{lstlisting}

好像有什么不对的地方? 假定我们对*赋给整数或者字符串值, 那做乘法时怎么办? 不管怎么说, * 总是乘法呀? 答案简单极了。Lisp中函数的角色十分特殊, 函数也是一种数据类型, 就像整数和字符串一样, 因此可以把它赋值给符号。乘法函数Lisp的内置函数, 默认赋给 * , 你可以把其他函数赋值给*, 那样*就不代表乘法了。你也可以把这函数的值存到另外的变量里。我们再用伪码来说明一下:
\begin{lstlisting}[language=lisp]
    * (3,4)          // 3乘4, 结果是12
    set(temp, * )    // 把 * 的值, 也就是乘法函数, 赋值给temp
    set( * , 3)       // 把3赋予 *
    * (3,4)          // 错误的表达式, * 不再是乘法, 而是数值3
    temp(3,4)       // temp是乘法函数, 所以此表达式的值为3乘4等于12
    set(* , temp)    // 再次把乘法函数赋予 *
    * (3,4)          // 3乘4等于12
\end{lstlisting}

再古怪一点, 把减号的值赋给加号:
\begin{lstlisting}[language=lisp]
    set(+, - )       // 减号( - )是内置的减法函数
    +(5, 4)         // 加号(+)现在是代表减法函数, 结果是5减4等于1
\end{lstlisting}

这只是举例子, 我还没有详细讲函数。Lisp中的函数是一种数据类型, 和整数, 字符串,符号等等一样。一个函数并不必然有一个名字, 这和C++或者Java语言的情形很不相同。在这里函数自己代表自己。事实上它是一个指向代码块的指针, 附带有一些其他信息(例如一组参数变量)。只有在把函数赋予其他符号时, 它才具有了名字, 就像把一个数值或字符串赋予变量一样的道理。你可以用一个内置的专门用于创建函数的函数来创建函数,然后把它赋值给符号fn, 用伪码来表示就是:
\begin{lstlisting}[language=lisp]
    fn [a]
    {
        return * (a, 2);
    }
\end{lstlisting}

这段代码返回一个具有一个参数的函数, 函数的功能是计算参数乘2的结果。这个函数还没有名字, 你可以把此函数赋值给别的符号:
\begin{lstlisting}[language=lisp]
    set(times-two, fn [a] {return * (a, 2)})
\end{lstlisting}

我们现在可以这样调用这个函数:
\begin{lstlisting}[language=lisp]
    time-two(5)         // 返回10
\end{lstlisting}

我们先跳过符号和函数, 讲一讲表。什么是表? 你也许已经听过好多相关的说法。表, 一言以蔽之, 就是把类似XML那样的数据块, 用s表达式来表示。表用一对括号括住, 表中元素以空格分隔, 表可以嵌套。例如(这回我们用真正的Lisp语法, 注意用分号表示注释):
\begin{lstlisting}[language=lisp]
    ()                      ; 空表
    (1)                     ; 含一个元素的表
    (1 "test")              ; 两元素表, 一个元素是整数1, 另一个是字符串
    (test "hello")          ; 两元素表, 一个元素是符号, 另一个是字符串
    (test (1 2) "hello")    ; 三元素表, 一个符号test, 一个含有两个元素1和2的
                            ; 表, 最后一个元素是字符串
\end{lstlisting}

当Lisp系统遇到这样的表时, 它所做的, 和Ant处理XML数据所做的, 非常相似, 那就是试图执行它们。其实, Lisp源码就是特定的一种表, 好比Ant源码是一种特定的XML一样。Lisp执行表的顺序是这样的, 表的第一个元素当作函数, 其他元素当作函数的参数。如果其中某个参数也是表, 那就按照同样的原则对这个表求值, 结果再传递给最初的函数作为参数。这就是基本原则。我们看一下真正的代码:
\begin{lstlisting}[language=lisp]
    ( * 3 4)                 ; 相当于前面列举过的伪码 * (3,4), 即计算3乘4
    (times-two 5)           ; 返回10, times-two按照前面的定义是求参数的2倍
    (3 4)                   ; 错误, 3不是函数
    (time-two)              ; 错误, times-two要求一个参数
    (times-two 3 4)         ; 错误, times-two只要求一个参数
    (set + -)               ; 把减法函数赋予符号+
    (+ 5 4)                 ; 依据上一句的结果, 此时+表示减法, 所以返回1
    ( * 3 (+ 2 2))           ; 2+2的结果是4, 再乘3, 结果是12
\end{lstlisting}

上述的例子中, 所有的表都是当作代码来处理的。怎样把表当作数据来处理呢? 同样的,设想一下, Ant是把XML数据当作自己的参数。在Lisp中, 我们给表加一个前缀'来表示数据。
\begin{lstlisting}[language=lisp]
    (set test '(1 2))       ; test的值为两元素表
    (set test (1 2))        ; 错误, 1不是函数
    (set test '( * 3 4))     ; test的值是三元素表, 三个元素分别是 *, 3, 4
\end{lstlisting}

我们可以用一个内置的函数head来返回表的第一个元素, tail函数来返回剩余元素组成的表。
\begin{lstlisting}[language=lisp]
    (head '( * 3 4))         ; 返回符号 *
    (tail '( * 3 4))         ; 返回表(3 4)
    (head (tal '( * 3 4)))   ; 返回3
    (head test)             ; 返回 *
\end{lstlisting}

你可以把Lisp的内置函数想像成Ant的任务。差别在于, 我们不用在另外的语言中扩展Lisp(虽然完全可以做得到), 我们可以用Lisp自己来扩展自己, 就像上面举的times-two函数的例子。Lisp的内置函数集十分精简, 只包含了十分必要的部分。剩下的函数都是作为标准库来实现的。
\section{Lisp宏}
\label{sec-8}

我们已经看到, 元编程在一个类似jsp的模板引擎方面的应用。我们通过简单的字符串处理来生成代码。但是我们可以做的更好。我们先提一个问题, 怎样写一个工具, 通过查找目录结构中的源文件来自动生成Ant脚本。

用字符串处理的方式生成Ant脚本是一种简单的方式。当然, 还有一种更加抽象, 表达能力更强, 扩展性更好的方式, 就是利用XML库在内存中直接生成XML节点, 这样的话内存中的节点就可以自动序列化成为字符串。不仅如此, 我们的工具还可以分析这些节点, 对已有的XML文件做变换。通过直接处理XML节点。我们可以超越字符串处理, 使用更高层次的概念, 因此我们的工作就会做的更快更好。

我们当然可以直接用Ant自身来处理XML变换和制作代码生成工具。或者我们也可以用Lisp来做这项工作。正像我们以前所知的, 表是Lisp内置的数据结构, Lisp含有大量的工具来快速有效的操作表(head和tail是最简单的两个)。而且, Lisp没有语义约束, 你可以构造任何数据结构, 只要你原意。

Lisp通过宏(macro)来做元编程。我们写一组宏来把任务列表(to-do list)转换为专用领域语言。

回想一下上面to-do list的例子, 其XML的数据格式是这样的:
\begin{lstlisting}[language=xml]
    <todo name = "housework">
        <item priority = "high">Clean the hose</item>
        <item priority = "medium">Wash the dishes</item>
        <item priority = "medium">Buy more soap</item>
    </todo>
\end{lstlisting}

相应的s表达式是这样的:
\begin{lstlisting}[language=lisp]
    (todo "housework"
        (item (priority high) "Clean the house")
        (item (priority medium) "Wash the dishes")
        (item (priority medium) "Buy more soap"))
\end{lstlisting}

假设我们要写一个任务表的管理程序, 把任务表数据存到一组文件里, 当程序启动时, 从文件读取这些数据并显示给用户。在别的语言里(比如说Java), 这个任务该怎么做? 我们会解析XML文件, 从中得出任务表数据, 然后写代码遍历XML树, 再转换为Java的数据结构(老实讲, 在Java里解析XML真不是件轻松的事情), 最后再把数据展示给用户。现在如果用Lisp, 该怎么做?

假定要用同样思路的化, 我们大概会用Lisp库来解析XML。XML对我们来说就是一个Lisp的表(s表达式), 我们可以遍历这个表, 然后把相关数据提交给用户。可是, 既然我们用Lisp, 就根本没有必要再用XML格式保存数据, 直接用s表达式就好了, 这样就没有必要做转换了。我们也用不着专门的解析库, Lisp可以直接在内存里处理s表达式。注意, Lisp编译器和.net编译器一样, 对Lisp程序来说, 在运行时总是随时可用的。

但是还有更好的办法。我们甚至不用写表达式来存储数据, 我们可以写宏, 把数据当作代码来处理。那该怎么做呢? 真的简单。回想一下, Lisp的函数调用格式:
\begin{lstlisting}[language=lisp]
    (function-name arg1 arg2 arg3)
\end{lstlisting}

其中每个参数都是s表达式, 求值以后, 传递给函数。如果我们用(+ 4 5)来代替arg1, 那么, 程序会先求出结果, 就是9, 然后把9传递给函数。宏的工作方式和函数类似。主要的差别是, 宏的参数在代入时不求值。
\begin{lstlisting}[language=lisp]
  (macro-name (+ 4 5))
\end{lstlisting}

这里, (+ 4 5)作为一个表传递给宏, 然后宏就可以任意处理这个表, 当然也可以对它求值。宏的返回值是一个表, 然后有程序作为代码来执行。宏所占的位置, 就被替换为这个结果代码。我们可以定义一个宏把数据替换为任意代码, 比方说, 替换为显示数据给用户的代码。

这和元编程, 以及我们要做的任务表程序有什么关系呢? 实际上, 编译器会替我们工作, 调用相应的宏。我们所要做的, 仅仅是创建一个把数据转换为适当代码的宏。

例如, 上面曾经将过的C的求三次方的宏, 用Lisp来写是这样子:
\begin{lstlisting}[language=lisp]
  (defmacro triple (x)
        `(+ ~x ~x ~x))
\end{lstlisting}

(译注: 在Common Lisp中, 此处的单引号应当是反单引号, 意思是对表不求值, 但可以对表中某元素求值, 记号\textasciitilde{}表示对元素x求值, 这个求值记号在Common Lisp中应当是逗号。反单引号和单引号的区别是, 单引号标识的表, 其中的元素都不求值。这里作者所用的记号是自己发明的一种Lisp方言Blaise, 和common lisp略有不同, 事实上, 发明方言是lisp高手独有的乐趣, 很多狂热分子都热衷这样做。比如Paul Graham就发明了ARC, 许多记号比传统的Lisp简洁得多, 显得比较现代)

单引号的用处是禁止对表求值。每次程序中出现triple的时候, 
\begin{lstlisting}[language=lisp]
    (triple 4)
\end{lstlisting}

都会被替换成:
\begin{lstlisting}[language=lisp]
    (+ 4 4 4)
\end{lstlisting}

我们可以为任务表程序写一个宏, 把任务数据转换为可执行码, 然后执行。假定我们的输出是在控制台:
\begin{lstlisting}[language=lisp]
    (defmacro item (priority note)
        `(block 
            (print stdout tab "Prority: " ~(head (tail priority)) endl)
            (print stdout tab "Note: " ~note endl endl)))
\end{lstlisting}

我们创造了一个非常小的有限的语言来管理嵌在Lisp中的任务表。这个语言只用来解决特定领域的问题, 通常称之为DSLs(特定领域语言, 或专用领域语言)。
\section{特定领域语言}
\label{sec-9}

本文谈到了两个特定领域语言, 一个是Ant, 处理软件构造。一个是没起名字的, 用于处理任务表。两者的差别在于, Ant是用XML, XML解析器, 以及Java语言合在一起构造出来的。而我们的迷你语言则完全内嵌在Lisp中, 只消几分钟就做出来了。

我们已经说过了DSL的好处, 这也就是Ant用XML而不直接用Java的原因。如果使用Lisp, 我们可以任意创建DSL, 只要我们需要。我们可以创建用于网站程序的DSL, 可以写多用户游戏, 做固定收益贸易(fixed income trade), 解决蛋白质折叠问题, 处理事务问题, 等等。我们可以把这些叠放在一起, 造出一个语言, 专门解决基于网络的贸易程序, 既有网络语言的优势, 又有贸易语言的好处。每天我们都会收获这种方法带给我们的益处, 远远超过Ant所能给予我们的。

用DSL解决问题, 做出的程序精简, 易于维护, 富有弹性。在Java里面, 我们可以用类来处理问题。这两种方法的差别在于, Lisp使我们达到了一个更高层次的抽象, 我们不再受语言解析器本身的限制, 比较一下用Java库直接写的构造脚本和用Ant写的构造脚本其间的差别。同样的, 比较一下你以前所做的工作, 你就会明白Lisp带来的好处。
\section{接下来}
\label{sec-10}

学习Lisp就像战争中争夺山头。尽管在电脑科学领域, Lisp已经算是一门古老的语言, 直到现在仍然很少有人真的明白该怎样给初学者讲授Lisp。尽管Lisp老手们尽了很大努力,今天新手学习Lisp仍然是困难重重。好在现在事情正在发生变化, Lisp的资源正在迅速增加, 随着时间推移, Lisp将会越来越受关注。

Lisp使人超越平庸, 走到前沿。学会Lisp意味着你能找到更好的工作, 因为聪明的雇主会被你与众不同的洞察力所打动。学会Lisp也可能意味着明天你可能会被解雇, 因为你总是强调, 如果公司所有软件都用Lisp写, 公司将会如何卓越, 而这些话你的同事会听烦的。Lisp值得努力学习吗? 那些已经学会Lisp的人都说值得, 当然, 这取决于你的判断。

你的看法呢?

这篇文章写写停停, 用了几个月才最终完成。如果你觉得有趣, 或者有什么问题, 意见或建议, 请给我发邮件coffeemug@gmail.com,我会很高兴收到你的反馈。 
% Emacs 24.3.50.1 (Org mode 8.2.5h)
\end{CJK}
\end{document}