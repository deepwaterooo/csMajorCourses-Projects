% Created 2014-02-25 二 23:00
\documentclass[11pt]{article}
\usepackage[utf8]{inputenc}
\usepackage[T1]{fontenc}
\usepackage{fixltx2e}
\usepackage{graphicx}
\usepackage{longtable}
\usepackage{float}
\usepackage{wrapfig}
\usepackage{rotating}
\usepackage[normalem]{ulem}
\usepackage{amsmath}
\usepackage{textcomp}
\usepackage{marvosym}
\usepackage{wasysym}
\usepackage{amssymb}
\usepackage{hyperref}
\tolerance=1000
\usepackage{CJKutf8}
\usepackage{listings}
\begin{CJK}{UTF8}{gbsn}
\author{jenny}
\date{\today}
\title{oo}
\hypersetup{
  pdfkeywords={},
  pdfsubject={},
  pdfcreator={Emacs 24.3.50.1 (Org mode 8.2.5h)}}
\begin{document}

\maketitle
\tableofcontents

\usepackage{geometry}
\geometry{left=0cm,right=0cm,top=0cm,bottom=0cm}

\lstset\{language=c++,
numbers=left, 
numberstyle=\tiny,
basicstyle=\ttfamily\small,
tabsize=4,
frame=none, 
escapeinside=``, 
extendedchars=false
\}

修练8年C++面向对象程序设计之体会
2005-04-29 09:58  出处:  作者:林锐  责任编辑:xietaoming 

  六年前,我刚热恋“面向对象”(Object-Oriented)时,一口气记住了近十个定义。六年后,我从几十万行程序中滚爬出来准备写点心得体会时,却无法解释什么是“面向对象”,就象说不清楚什么是数学那样。软件工程中的时髦术语“面向对象分析”和“面向对象设计”,通常是针对“需求分析”和“系统设计”环节的。“面向对象”有几大学派,就象如来佛、上帝和真主用各自的方式定义了这个世界,并留下一堆经书来解释这个世界。

  有些学者建议这样找“对象”:分析一个句子的语法,找出名词和动词,名词就是对象,动词则是对象的方法(即函数)。

  当年国民党的文人为了对抗毛泽东的《沁园春·雪》,特意请清朝遗老们写了一些对仗工整的诗,请蒋介石过目。老蒋看了气得大骂:“娘希匹,全都有一股棺材里腐尸的气味。”我看了几千页的软件工程资料,终于发现自己有些“弱智”,无法理解“面向对象”的理论,同时醒悟到“编程是硬道理。”

  面向对象程序设计语言很多,如Smalltalk、Ada、Eiffel、Object Pascal、Visual Basic、C++等等。C++语言最讨人喜欢,因为它兼容C 语言,并且具备C 语言的性能。近几年,一种叫Java 的纯面向对象语言红极一时,不少人叫喊着要用Java 革C++的命。我认为Java 好比是C++的外甥,虽然不是直接遗传的,但也几分象样。外甥在舅舅身上玩耍时洒了一泡尿,俩人不该为此而争吵。

  关于C++程序设计的书藉非常多,本章不讲C++的语法,只讲一些小小的编程道理。如果我能早几年明白这些小道理,就可以大大改善数十万行程序的质量了。

\section{C++面向对象程序设计的重要概念}
\label{sec-1}

  早期革命影片里有这样一个角色,他说:“我是党代表,我代表党,我就是党。”后来他给同志们带来了灾难。

  会用C++的程序员一定懂得面向对象程序设计吗?

  不会用C++的程序员一定不懂得面向对象程序设计吗?

  两者都未必。就象坏蛋入党后未必能成为好人,好人不入党未必变成坏蛋那样。

  我不怕触犯众怒地说句大话:“C++没有高手,C 语言才有高手。”在用C 和C++编程8年之后,我深深地遗憾自己不是C 语言的高手,更遗憾没有人点拨我如何进行面向对象程序设计。我和很多C++程序员一样,在享用到C++语法的好处时便以为自己已经明白了面向对象程序设计。就象挤掉牙膏卖牙膏皮那样,真是暴殄天物呀。

  人们不懂拼音也会讲普通话,如果懂得拼音则会把普通话讲得更好。不懂面向对象程序设计也可以用C++编程,如果懂得面向对象程序设计则会把C++程序编得更好。本节讲述三个非常基础的概念:“类与对象”、“继承与组合”、“虚函数与多态”。理解这些概念,有助于提高程序的质量,特别是提高“可复用性”与“可扩充性”。

\subsection{类与对象}
\label{sec-1-1}

  对象(Object)是类(Class)的一个实例(Instance)。如果将对象比作房子,那么类就是房子的设计图纸。所以面向对象程序设计的重点是类的设计,而不是对象的设计。类可以将数据和函数封装在一起,其中函数表示了类的行为(或称服务)。类提供关键字public、protected 和private 用于声明哪些数据和函数是公有的、受保护的或者是私有的。

  这样可以达到信息隐藏的目的,即让类仅仅公开必须要让外界知道的内容,而隐藏其它一切内容。我们不可以滥用类的封装功能,不要把它当成火锅,什么东西都往里扔。

  类的设计是以数据为中心,还是以行为为中心?

  主张“以数据为中心”的那一派人关注类的内部数据结构,他们习惯上将private 类型的数据写在前面,而将public 类型的函数写在后面,如表8.1(a)所示。

  主张“以行为为中心”的那一派人关注类应该提供什么样的服务和接口,他们习惯上将public 类型的函数写在前面,而将private 类型的数据写在后面,如表8.1(b)所示。

  很多C++教课书主张在设计类时“以数据为中心”。我坚持并且建议读者在设计类时“以行为为中心”,即首先考虑类应该提供什么样的函数。Microsoft 公司的COM 规范的核心是接口设计,COM 的接口就相当于类的公有函数[Rogerson 1999]。在程序设计方面,咱们不要怀疑Microsoft 公司的风格。

  设计孤立的类是比较容易的,难的是正确设计基类及其派生类。因为有些程序员搞不清楚“继承”(Inheritance)、“组合”(Composition)、“多态”( Polymorphism)这些概念。
\subsection{继承与组合}
\label{sec-1-2}

  如果A 是基类,B 是A 的派生类,那么B 将继承A 的数据和函数。示例程序如下:
\begin{lstlisting}[language=c++]
class A {
public:
    void Func1(void);
    void Func2(void);
};

class B : public A {
public:
    void Func3(void);
    void Func4(void);
};

// Example
main() {
    B b; // B的一个对象
    b.Func1(); // B 从A 继承了函数Func1
    b.Func2(); // B 从A 继承了函数Func2
    b.Func3();
    b.Func4();
}
\end{lstlisting}

  这个简单的示例程序说明了一个事实:C++的“继承”特性可以提高程序的可复用性。正因为“继承”太有用、太容易用,才要防止乱用“继承”。我们要给“继承”立一些使用规则:

\subsubsection{  如果类A 和类B 毫不相关,不可以为了使B 的功能更多些而让B 继承A 的功能。}
\label{sec-1-2-1}

  不要觉得“不吃白不吃”,让一个好端端的健壮青年无缘无故地吃人参补身体。
\subsubsection{  如果类B 有必要使用A 的功能,则要分两种情况考虑:}
\label{sec-1-2-2}

\begin{enumerate}
\item   若在逻辑上B 是A 的“一种”(a kind of ),则允许B 继承A 的功能。如男人(Man)是人(Human)的一种,男孩(Boy)是男人的一种。那么类Man 可以从类Human 派生,类Boy 可以从类Man 派生。示例程序如下:
\label{sec-1-2-2-1}
\begin{lstlisting}[language=c++]
class Human {
    …
};

class Man : public Human {
    …
};

class Boy : public Man {
    …
};
\end{lstlisting}
\item   若在逻辑上A 是B 的“一部分”(a part of),则不允许B 继承A 的功能,而是要用A和其它东西组合出B。例如眼(Eye)、鼻(Nose)、口(Mouth)、耳(Ear)是头(Head)的一部分,所以类Head 应该由类Eye、Nose、Mouth、Ear 组合而成,不是派生而成。示例程序如下:
\label{sec-1-2-2-2}
\begin{lstlisting}[language=c++]
class Eye {
public:
    void Look(void);
};

class Nose {
public:
    void Smell(void);
};

class Mouth {
public:
    void Eat(void);
};

class Ear {
public:
    void Listen(void);
};

// 正确的设计,冗长的程序
class Head {
public:
    void Look(void) { m_eye.Look(); }
    void Smell(void) { m_nose.Smell(); }
    void Eat(void) { m_mouth.Eat(); }
    void Listen(void) { m_ear.Listen(); }
private:
    Eye m_eye;
    Nose m_nose;
    Mouth m_mouth;
    Ear m_ear;
};
\end{lstlisting}

  如果允许Head 从Eye、Nose、Mouth、Ear 派生而成,那么Head 将自动具有Look、Smell、Eat、Listen 这些功能:
\begin{lstlisting}[language=c++]
// 错误的设计
class Head : public Eye, public Nose, public Mouth, public Ear {
};
\end{lstlisting}

  上述程序十分简短并且运行正确,但是这种设计却是错误的。很多程序员经不起“继承”的诱惑而犯下设计错误。

  一只公鸡使劲地追打一只刚下了蛋的母鸡,你知道为什么吗?

  因为母鸡下了鸭蛋。

  本书3.3 节讲过“运行正确”的程序不见得就是高质量的程序,此处就是一个例证。
\end{enumerate}
\subsection{虚函数与多态}
\label{sec-1-3}

  除了继承外,C++的另一个优良特性是支持多态,即允许将派生类的对象当作基类的对象使用。如果A 是基类,B 和C 是A 的派生类,多态函数Test 的参数是A 的 指针。那么Test 函数可以引用A、B、C 的对象。示例程序如下:
\begin{lstlisting}[language=c++]
class A {
public:
    void Func1(void);
};

void Test(A *a) {
    a->Func1();
}

class B : public A {
    …
};

class C : public A {
    …
};

// Example
main() {
    A a;
    B b;
    C c;
    Test(&a);
    Test(&b);
    Test(&c);
};
\end{lstlisting}

  以上程序看不出“多态”有什么价值,加上虚函数和抽象基类后,“多态”的威力就显示出来了。

  C++用关键字virtual 来声明一个函数为虚函数,派生类的虚函数将(override)基类对应的虚函数的功能。示例程序如下:
\begin{lstlisting}[language=c++]
class A {
public:
    virtual void Func1(void){ cout<< “This is A::Func1 \n”}
};

void Test(A *a) {
    a->Func1();
}

class B : public A {
public:
    virtual void Func1(void){ cout<< “This is B::Func1 \n”}
};

class C : public A {
public:
    virtual void Func1(void){ cout<< “This is C::Func1 \n”}
};

// Example
main() {
    A a;
    B b;
    C c;
    Test(&a); // 输出This is A::Func1
    Test(&b); // 输出This is B::Func1
    Test(&c); // 输出This is C::Func1
};
\end{lstlisting}

  如果基类A 定义如下:
\begin{lstlisting}[language=c++]

class A {
public:
    virtual void Func1(void)=0;
};
\end{lstlisting}

  那么函数Func1 叫作纯虚函数,含有纯虚函数的类叫作抽象基类。抽象基类只管定义纯虚函数的形式,具体的功能由派生类实现。

  结合“抽象基类”和“多态”有如下突出优点:

  (1)应用程序不必为每一个派生类编写功能调用,只需要对抽象基类进行处理即可。这一
招叫“以不变应万变”,可以大大提高程序的可复用性(这是接口设计的复用,而不是代码实现的复用)。

  (2)派生类的功能可以被基类指针引用,这叫向后兼容,可以提高程序的可扩充性和可维护性。以前写的程序可以被将来写的程序调用不足为奇,但是将来写的程序可以被以前写的程序调用那可了不起。
\section{良好的编程风格}
\label{sec-2}

  内功深厚的武林高手出招往往平淡无奇。同理,编程高手也不会用奇门怪招写程序。良好的编程风格是产生高质量程序的前提。

\subsection{命名约定}
\label{sec-2-1}

  有不少人编程时用拼音给函数或变量命名,这样做并不能说明你很爱国,却会让用此程序的人迷糊(很多南方人不懂拼音,我就不懂)。程序中的英文一般不会太复杂,用词要力求准确。匈牙利命名法是Microsoft 公司倡导的[Maguire 1993],虽然很烦琐,但用习惯了也就成了自然。没有人强迫你采用何种命名法,但有一点应该做到:自己的程序命名必须一致。

 以下是我编程时采用的命名约定:

  (1)宏定义用大写字母加下划线表示,如MAX$_{\text{LENGTH;}}$

  (2)函数用大写字母开头的单词组合而成,如SetName, GetName ;

  (3)指针变量加前缀p,如*pNode ;

  (4)BOOL 变量加前缀b,如bFlag ;

  (5)int 变量加前缀i,如iWidth ;

  (6)float 变量加前缀f,如fWidth ;

  (7)double 变量加前缀d,如dWidth ;

  (8)字符串变量加前缀str,如strName ;

  (9)枚举变量加前缀e,如eDrawMode ;

  (10)类的成员变量加前缀m$_{\text{,如m}}$$_{\text{strName}}$, m$_{\text{iWidth}}$ ;

  对于int, float, double 型的变量,如果变量名的含义十分明显,则不加前缀,避免烦琐。如用于循环的int 型变量i,j,k ;float 型的三维坐标(x,y,z)等。
\subsection{使用断言}
\label{sec-2-2}

  程序一般分为Debug 版本和Release 版本,Debug 版本用于内部调试,Release 版本发行给用户使用。断言assert 是仅在Debug 版本起作用的宏,它用于检查“不应该”发生的情况。以下是一个内存复制程序,在运行过程中,如果assert 的参数为假,那么程序就会中止(一般地还会出现提示对话,说明在什么地方引发了assert)。
\begin{lstlisting}[language=c++]
//复制不重叠的内存块
void memcpy(void *pvTo, void *pvFrom, size_t size) {
    void *pbTo = (byte *) pvTo;
    void *pbFrom = (byte *) pvFrom;
    assert( pvTo != NULL && pvFrom != NULL );
    while(size - - > 0 )
        *pbTo + + = *pbFrom + + ;
    return (pvTo);
}
\end{lstlisting}

  assert 不是一个仓促拼凑起来的宏,为了不在程序的Debug 版本和Release 版本引起差别,assert 不应该产生任何副作用。所以assert 不是函数,而是宏。程序员可以把assert 看成一个在任何系统状态下都可以安全使用的无害测试手段。

  很少有比跟踪到程序的断言,却不知道该断言的作用更让人沮丧的事了。你化了很多时间,不是为了排除错误,而只是为了弄清楚这个错误到底是什么。有的时候,程序员偶尔还会设计出有错误的断言。所以如果搞不清楚断言检查的是什么,就很难判断错误是出现在程序中,还是出现在断言中。幸运的是这个问题很好解决,只要加上清晰的注释即可。这本是显而易见的事情,可是很少有程序员这样做。这好比一个人在森林里,看到树上钉着一块“危险”的大牌子。但危险到底是什么?树要倒?有废井?有野兽?除非告诉人们“危险”是什么,否则这个警告牌难以起到积极有效的作用。难以理解的断言常常被程序员忽略,甚至被删除。[Maguire 1993]

  以下是使用断言的几个原则:

  (1)使用断言捕捉不应该发生的非法情况。不要混淆非法情况与错误情况之间的区别,后者是必然存在的并且是一定要作出处理的。

  (2)使用断言对函数的参数进行确认。

  (3)在编写函数时,要进行反复的考查,并且自问:“我打算做哪些假定?”一旦确定了的
假定,就要使用断言对假定进行检查。

  (4)一般教科书都鼓励程序员们进行防错性的程序设计,但要记住这种编程风格会隐瞒错误。当进行防错性编程时,如果“不可能发生”的事情的确发生了,则要使用断言进行报警。
\subsection{new、delete 与指针}
\label{sec-2-3}

  在C++中,操作符new 用于申请内存,操作符delete 用于释放内存。在C 语言中,函数malloc 用于申请内存,函数free 用于释放内 存。由于C++兼容C 语言,所以new、delete、malloc、free 都有可能一起使用。new 能比malloc 干更多的事,它可以申请对象的内存,而malloc 不能。C++和C 语言中的指针威猛无比,用错了会带来灾难。对于一个指针p,如果是用new申请的内存,则必须用delete 而不能用free 来释放。如果是用malloc 申请的内存,则必须用free 而不能用delete 来释放。在用delete 或用free 释放p 所指的内存后,应该马上显式地将p 置为NULL,以防下次使用p 时发生错误。示例程序如下:
\begin{lstlisting}[language=c++]
void Test(void) {
    float *p;
    
    p = new float[100];
    if(p==NULL) return;
    …// do something
        delete p;
    p=NULL; // 良好的编程风格
    
    // 可以继续使用p
    p = new float[500];
    if(p==NULL) return;
    …// do something else
        delete p;
    p=NULL;
}
\end{lstlisting}

  我们还要预防“野指针”,“野指针”是指向“垃圾”内存的指针,主要成因有两种:

  (1)指针没有初始化。

  (2)指针指向已经释放的内存,这种情况最让人防不胜防,示例程序如下:
\begin{lstlisting}[language=c++]
class A {
public:
    void Func(void){…}
};

void Test(void) {
    A *p; {
        A a;
        p = &a; // 注意a 的生命期
    }
    p->Func(); // p 是“野指针”,程序出错
}
\end{lstlisting}
\subsection{使用const}
\label{sec-2-4}

  在定义一个常量时,const 比\#define 更加灵活。用const 定义的常量含有数据类型,该常量可以参与逻辑运算。例如:
\begin{lstlisting}[language=c++]
const int LENGTH = 100; // LENGTH 是int 类型
const float MAX=100; // MAX 是float 类型
#define LENGTH 100 // LENGTH 无类型
#define MAX 100 // MAX 无类型
\end{lstlisting}

  除了能定义常量外,const 还有两个“保护”功能:

\subsubsection{  一、强制保护函数的参数值不发生变化}
\label{sec-2-4-1}

  以下程序中,函数f 不会改变输入参数name 的值,但是函数g 和h 都有可能改变name的值。
\begin{lstlisting}[language=c++]
void f(String s); // pass by value
void g(String &s); // pass by referance
void h(String *s); // pass by pointer

main() {
    String name=“Dog”;
    f(name); // name 的值不会改变
    g(name); // name 的值可能改变
    h(name); // name 的值可能改变
}
\end{lstlisting}

  对于一个函数而言,如果其‘\&’或‘*’类型的参数只作输入用,不作输出用,那么应当在该参数前加上const,以确保函数的代码不会改变该参数的值(如果改变了该参数的值,编译器会出现错误警告)。因此上述程序中的函数g 和h 应该定义成:
\begin{lstlisting}[language=c++]
void g(const String &s);
void h(const String *s);
\end{lstlisting}
\subsubsection{  二、强制保护类的成员函数不改变任何数据成员的值}
\label{sec-2-4-2}

  以下程序中,类stack 的成员函数Count 仅用于计数,为了确保Count 不改变类中的任何数据成员的值,应将函数Count 定义成const 类型。
\begin{lstlisting}[language=c++]
class Stack {
public:
    void push(int elem);
    void pop(void);
    int Count(void) const; // const 类型的函数
private:
    int num;
    int data[100];
};

int Stack::Count(void) const {
    ++ num; // 编译错误,num 值发生变化
    pop(); // 编译错误,pop 将改变成员变量的值
    return num;
}
\end{lstlisting}
\subsection{其它建议}
\label{sec-2-5}

 (1)不要编写一条过分复杂的语句,紧凑的C++/C 代码并不见到能得到高效率的机器代码,却会降低程序的可理解性,程序出错误的几率也会提高。

  (2)不要编写集多种功能于一身的函数,在函数的返回值中,不要将正常值和错误标志混在一起。

  (3)不要将BOOL 值TRUE 和FALSE 对应于1 和0 进行编程。大多数编程语言将FALSE定义为0,任何非0 值都是TRUE。Visual C++将TRUE 定义为1,而Visual Basic 则将TRUE定义为-1。示例程序如下:

\begin{lstlisting}[language=c++]
BOOL flag;

if(flag)       { // do something } // 正确的用法
if(flag==TRUE) { // do something } // 危险的用法
if(flag==1)    { // do something } // 危险的用法
if(!flag)      { // do something } // 正确的用法
if(flag==FALSE){ // do something } // 不合理的用法
if(flag==0)    { // do something } // 不合理的用法
\end{lstlisting}

  (4)小心不要将“= =”写成“=”,编译器不会自动发现这种错误。

  (5)不要将123 写成0123,后者是八进制的数值。

  (6)将自己经常犯的编程错误记录下来,制成表格贴在计算机旁边。
\section{小结}
\label{sec-3}

  C++/C 程序设计如同少林寺的武功一样博大精深,我练了8 年,大概只学到二三成。所以无论什么时候,都不要觉得自己的编程水平天下第一,看到别人好的技术和风格,要虚心学习。本章的内容少得可怜,就象口渴时只给你一颗杨梅吃,你一定不过瘾。我借花献佛,推荐一本好书:Marshall P. Cline 著的《C++ FAQs》[Cline 1995]。你看了后一定会赞不绝口。会编写C++/C 程序,不要因此得意洋洋,这只是程序员基本的技能要求而已。如果把系统分析和系统设计比作“战略决策”,那么编程充其量只是“战术”。如果指挥官是个大笨蛋,士兵再勇敢也会吃败仗。所以我们程序员不要只把眼光盯在程序上,要让自己博学多才。我们应该向北京胡同里的小孩们学习,他们小小年纪就能指点江山,评论世界大事。
% Emacs 24.3.50.1 (Org mode 8.2.5h)
\end{document}