 
2.	Overall Description
2.1 Product Perspective

This is a new software product designed for university students to make appointments with instructors. The on campus instructors and the unviersity students are the major two group of users who will benefit from this system. In case both instructors and students may need some extra help, department sectaries are a third class of users to help modify the weekly schedules of instructors and help rearrange schedules when it comes to be the rush periods (near difficult homework deadlines, or exams are coming) and many students want to get appointed but all available periods are occupied already. 

2.2 Product Functions

2.2.1 Instructor's Functions
From instructor's side, the main functions include: post weekly scheduling, modify schedule, modify appointment, delete/cancel student's appointment and add(rearrange) appointments. 
2.2.1.1 Post Schedule
2.2.1.2 Modify Schedule
2.2.1.3 Modify Appointments
2.2.1.3 Add Appointments
2.2.1.4 Delete Appointments
     
2.2.2 Student's Functions
From Student's side, the main functions include: add(schedule) appointment, delete(cancel) appointment, receiving automatic emails once add/delete/(be modified by instructor/secretaries) any appointments.
2.2.2.1 Add Appointments
2.2.2.2 Delete Appointments
     
2.2.3 Secetery's Functions
From Secetary's side, the main functions include: modify instructor's schedule, add/delete appointments
2.2.3.1 Modify Schedule
2.2.3.2 Add Appointmenst
2.2.3.3 Delete Appointments

2.2.4 System Functions
System functions are the main program that run and maintain the Web Appointment Scheduling System. The main funciton includes display weekly schedule, update Schedule and send out email to both Instructors and students. 
2.2.4.1 display weekly schedule
2.2.4.2 update Schedule
2.2.4.3 send out email

2.3 User Classes and Characteristics

There are three main user groups, and according to group's operation and software interface, each user group function should at least have one sperate Python module/class definition or extended class member functions so that the software implementation will include at least three major user classes, and each class would need to define/(extend to have) all possible/potential member functions which makes all three user groups' operations logical manipulatable and executable. 

Followed are the user classes, member functions and possible/potential arguments:

2.2.1 Instructor's Functions: Class Instructor(){}
2.2.1.1 Post Schedule:      start date, end date, 
2.2.1.2 Modify Schedule:    date, schetype, 
2.2.1.3 Modify Appointments: date, appflag, applenpre, starttime
2.2.1.3 Add Appointments:   date, starttime, applenpre, appflag
2.2.1.4 Delete Appointments: date, starttime

2.2.2 Student's Functions: Class Student(){}
2.2.2.1 Add Appointments:    date, starttime, applenpre, msg
2.2.2.2 Delete Appointments: date, starttime, applenpre, msg

2.2.3 Secetery's Functions: Class Secretary(){}
2.2.3.1 Modify Schedule:     date, schetype, 
2.2.3.2 Add Appointmenst:    date, starttime, applenpre, msg
2.2.3.3 Delete Appointments: date, starttime, applenpre, msg

2.4 Operating Environment

This software will be Window's, Linux and potentially mac cooperate. 
The software requirements would be internet explore, or firefox, and internet surfing software needed.
For access to the system, valid university/campus email address and password are required for accessing the software system. 

2.5 Design and Implementation Constraints

Right now, we assume/suppose we would get university account system safely accessible for our software. This is only the initial ideas without feasibility consideration and research yet, so we are not sure yet we can completely achieve this goal. 
so far no server accessibility consideration yet, we are not sure if we have available resource for this software to be implemented to use yet. 
we assume we are going to use university student email account or employee email account as the user check. But we have not consider any potential security issues yet. there may exist some restrictions from the security point of view. 

Both Instructors and students could have used different browsers like IE, Firefox and chrome in different operating systems, our system functions must be compatible with all these browsers and operating Systems. But right now, we focus on only one most commonly used browser and operating system. We may have to make necessary modifications in order to be user-friendly. Besides, it could be possible that our System is incompatible for certain operating System with specific browser.

2.6 User Documentation

There are certain rules for all the user groups. and there will be a simple user's guide/manual for each group displayed right behind the scheduling 2-dimensional table, which will least from the most important concerns/issues to least significant ones, so that every user within any group would easily catch the essentials and succcessfully satisfy their requirements. 

2.7 Assumptions and Dependencies

Right now, we assume/suppose we would get university account system safely accessible for our software. This is only the initial ideas without feasibility consideration and research yet, so we are not sure yet we can completely achieve this goal.

By designing and reviewing the software specifications so far, we assume the project goal is clear, the implementation language is feasible to achieve the software design goals, and we would be able to fair easily complete the goal.
 
But from the software application in everyday life, whether the software can be widely used by instructors or students will still depends on the logging systems, software sever availability and user performs. So far, we assume all users are normal, regular user for appointment and studying, solving problem propose, but potential malicous operations may be a threat for the software application. 



3.	External Interface Requirements

The only link to an external system is the link to VandalWeb username and password database to validate the authentication of our system. The editor believes that the VandalWeb username and password have complete information for all the users using our Web Appointment Scheduling System, if not larger. So it is a useful database for us to apply on our system. 

When the users typed in username and password, we will use the username as the key to search through the database, and compare the user-typed password with the password saved in VandalWeb database to see if they match. Authentication will be granted when they match, otherwise the user will have three chance to try other passwords before they get blocked for one day.  


3.1	User Interfaces
<Describe the logical characteristics of each interface between the software product and the users. This may include sample screen images, any GUI standards or product family style guides that are to be followed, screen layout constraints, standard buttons and functions (e.g., help) that will appear on every screen, keyboard shortcuts, error message display standards, and so on. Define the software components for which a user interface is needed. Details of the user interface design should be documented in a separate user interface specification.>

Student Interface: Week schedule/ Daily schedule with the following buttons:
Select, Confirm, Cancel, Memo.

Faculty Interface: Week schedule/ Daily schedule with the following buttons:
Select Starting time, End time; Select length of time slots; Confirm and publish schedule, Cancel, Block, Memo.

Secretary Interface: Week schedule/ Daily schedule with the following buttons:
Select Starting time, End time; Select length of time slots; Confirm and publish schedule, Cancel, Block, Memo.


3.2	Hardware Interfaces
<Describe the logical and physical characteristics of each interface between the software product and the hardware components of the system. This may include the supported device types, the nature of the data and control interactions between the software and the hardware, and communication protocols to be used.>
3.3	Software Interfaces
<Describe the connections between this product and other specific software components (name and version), including databases, operating systems, tools, libraries, and integrated commercial components. Identify the data items or messages coming into the system and going out and describe the purpose of each. Describe the services needed and the nature of communications. Refer to documents that describe detailed application programming interface protocols. Identify data that will be shared across software components. If the data sharing mechanism must be implemented in a specific way (for example, use of a global data area in a multitasking operating system), specify this as an implementation constraint.>
3.4	Communications Interfaces
<Describe the requirements associated with any communications functions required by this product, including e-mail, web browser, network server communications protocols, electronic forms, and so on. Define any pertinent message formatting. Identify any communication standards that will be used, such as FTP or HTTP. Specify any communication security or encryption issues, data transfer rates, and synchronization mechanisms.>

The software is associated with vandal-mail, web browser, vandalweb network server.
4.	System Features
<This template illustrates organizing the functional requirements for the product by system features, the major services provided by the product. You may prefer to organize this section by use case, mode of operation, user class, object class, functional hierarchy, or combinations of these, whatever makes the most logical sense for your product.>
4.1	System Feature 1
<Don't really say "System Feature 1." State the feature name in just a few words.>
4.1.1	Description and Priority
	<Provide a short description of the feature and indicate whether it is of High, Medium, or Low priority. You could also include specific priority component ratings, such as benefit, penalty, cost, and risk (each rated on a relative scale from a low of 1 to a high of 9).>
4.1.2	Stimulus/Response Sequences
	<List the sequences of user actions and system responses that stimulate the behavior defined for this feature. These will correspond to the dialog elements associated with use cases.>
4.1.3	Functional Requirements
	<Itemize the detailed functional requirements associated with this feature. These are the software capabilities that must be present in order for the user to carry out the services provided by the feature, or to execute the use case. Include how the product should respond to anticipated error conditions or invalid inputs. Requirements should be concise, complete, unambiguous, verifiable, and necessary. Use "TBD" as a placeholder to indicate when necessary information is not yet available.>
	
	<Each requirement should be uniquely identified with a sequence number or a meaningful tag of some kind.>
	
REQ-1:	
REQ-2:	
4.2	System Feature 2 (and so on)
5.	Other Nonfunctional Requirements
5.1	Performance Requirements
<If there are performance requirements for the product under various circumstances, state them here and explain their rationale, to help the developers understand the intent and make suitable design choices. Specify the timing relationships for real time systems. Make such requirements as specific as possible. You may need to state performance requirements for individual functional requirements or features.>

Right now, we assume/suppose we would get university account system safely accessible for our software. This is only the initial ideas without feasibility consideration and research yet, so we are not sure yet we can completely achieve this goal.

5.2	Safety Requirements
<Specify those requirements that are concerned with possible loss, damage, or harm that could result from the use of the product. Define any safeguards or actions that must be taken, as well as actions that must be prevented. Refer to any external policies or regulations that state safety issues that affect the product's design or use. Define any safety certifications that must be satisfied.>

Having not considering if we use student email account, if there will be any safety issues with it.
5.3	Security Requirements
<Specify any requirements regarding security or privacy issues surrounding use of the product or protection of the data used or created by the product. Define any user identity authentication requirements. Refer to any external policies or regulations containing security issues that affect the product. Define any security or privacy certifications that must be satisfied.>

But from the software application in everyday life, whether the software can be widely used by instructors or students will still depends on the logging systems, software sever availability and user performs. So far, we assume all users are normal, regular user for appointment and studying, solving problem propose, but potential malicous operations may be a threat for the software application. 

5.4	Software Quality Attributes
<Specify any additional quality characteristics for the product that will be important to either the customers or the developers. Some to consider are: adaptability, availability, correctness, flexibility, interoperability, maintainability, portability, reliability, reusability, robustness, testability, and usability. Write these to be specific, quantitative, and verifiable when possible. At the least, clarify the relative preferences for various attributes, such as ease of use over ease of learning.>

Will send of servay to all groups of users to get quality evaluation from users. 

5.5	Business Rules
<List any operating principles about the product, such as which individuals or roles can perform which functions under specific circumstances. These are not functional requirements in themselves, but they may imply certain functional requirements to enforce the rules.>

Initiative creation. No copying.
6.	Other Requirements
<Define any other requirements not covered elsewhere in the SRS. This might include database requirements, internationalization requirements, legal requirements, reuse objectives for the project, and so on. Add any new sections that are pertinent to the project.>
Appendix A: Glossary
<Define all the terms necessary to properly interpret the SRS, including acronyms and abbreviations. You may wish to build a separate glossary that spans multiple projects or the entire organization, and just include terms specific to a single project in each SRS.>
Appendix B: Analysis Models
<Optionally, include any pertinent analysis models, such as data flow diagrams, class diagrams, state-transition diagrams, or entity-relationship diagrams.>
Appendix C: To Be Determined List
<Collect a numbered list of the TBD (to be determined) references that remain in the SRS so they can be tracked to closure.>


\end{document}                          % The required last line